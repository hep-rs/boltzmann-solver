\documentclass{scrartcl}

\usepackage{mathtools}

\mathtoolsset{
  showonlyrefs,
}

\usepackage[compat=1.1.0]{tikz-feynman}
\usepackage{jpellis}

\usepackage{cleveref}

\title{Boltzmann Solver}
\author{Joshua P.~Ellis}

\begin{document}

\maketitle

\vfill

\tableofcontents

\clearpage

\section{Phase Space integration}%
\label{sec:phase_space_integration}

Given an \(n\) particle initial or final state, the phase space integration is
given by:
\begin{equation}
  \dd \Phi_n(q; p_1, \dots, p_n)
  \defeq \dd \Pi_1 \cdots \dd \Pi_n (2 \pi)^4
  \delta^4(p_1 + \cdots + p_n - q)
\end{equation}
where
\begin{equation}
  \dd \Pi
  \defeq \frac{g}{2 E} \ddbar^3{\vt p}
  = \frac{\vt p^2}{2 E} \frac{\dd \abs{\vt p} \dd \Omega}{(2\pi)^3}
  = \frac{\abs{\vt p}}{2} \frac{\dd E \dd \Omega}{(2 \pi)^3}
\end{equation}
where the last equality is achieved by noting that \(m^2 = E^2 - \abs{\vt p}^2\)
thereby implying that \(\abs{\vt p} \dd \abs{\vt p} \equiv E \dd E\).

\subsection{Two-Body Phase Space}%
\label{sec:two-body_phase_space}

The two-body phase space, \(\Phi_2\), is the smallest non-trivial space that can
be integrated over easily.

Defining the two-body phase space \(\Phi_2(q_{12}; p_1, p_2)\), where \(q_{12} =
(p_1 + p_2)\) is the 4-momentum shared between these two particle and \(s_{12}^2
\defeq q_{12}^2\) is the centre-of-mass energy of these two particles.
\begin{align}
  \dd \Phi_2(q_{12}; p_1, p_2)
   & = \dd \Pi_1 \dd \Pi_2 (2 \pi)^4 \delta^4(p_1 + p_2 - q_{12})                                                   \\
   & = \frac{g_1}{2 E_1} \ddbar^3{\vt p_1} \frac{g_2}{2 E_2} \ddbar^3{\vt p_2}
  (2\pi) \delta(E_1 + E_2 - E_{12}) (2\pi)^3 \delta^3(\vt p_1 + \vt p_2 - \vt q_{12})                               \\
  %  & = \frac{g_1 g_2}{4 (2 \pi)^2} \frac{\dd^3 \vt p_1}{E_1} \frac{\dd^3 \vt p_2}{E_2}
  % \delta(E_1 + E_2 - E_{12}) \delta^3(\vt p_1 + \vt p_2 - \vt q_{12})                                               \\
  \intertext{The integral over \(\vt p_2\) can be done trivially with the Dirac delta (note that after this integration, \(E_2\) no longer depends on \(\abs{\vt p_2}\) and now depends on \(\abs{\vt q_{12} - \vt p_1}\))}
   & = \frac{g_1 g_2}{16 \pi^2} \frac{\dd^3 \vt p_1}{E_1} \frac{1}{E_2}
  \delta(E_1 + E_2 - E_{12})
  \intertext{The final quantity is Lorentz invariant, so we may choose the centre-of-mass frame for the calculations in which \(p_1 = (E_1, \vt p)\), \(p_2 = (E_2, - \vt p)\), and \(q_{12} = (\sqrt{s_{12}}, \vt 0)\).}
   & = \frac{g_1 g_2}{16 \pi^2} \frac{\abs{\vt p}^2 \dd \abs{\vt p}}{E_1 E_2}
  \delta(E_1 + E_2 - \sqrt{s_{12}}) \dd \Omega
  \intertext{Given that \(E_1 = \sqrt{\abs{\vt p}^2 + m_1^2}\) and similarly for \(E_2\), the zeros of the Dirac delta can be found at \(\abs{\vt p} = \lambda^{\frac{1}{2}}(s_{12}, m_1^2, m_2^2) / 2 \sqrt{s_{12}}\):}
   & = \frac{g_1 g_2}{16 \pi^2} \frac{\abs{\vt p}^2 \dd \abs{\vt p}}{E_1 E_2}
  \left[ \ddfrac{(E_1 + E_2)}{\abs{\vt p}} \right]^{-1}
  \delta\left( \abs{\vt p} - \frac{\lambda^{\frac{1}{2}}(s_{12}, m_1^2, m_2^2)}{2 \sqrt{s_{12}}} \right) \dd \Omega \\
   & = \frac{g_1 g_2}{16 \pi^2} \dd \abs{\vt p} \frac{\abs{\vt p}^2 }{4 E_1 E_2}
  \left[ \frac{\abs{\vt p}}{E_1} + \frac{\abs{\vt p}}{E_2} \right]^{-1}
  \delta\left( \abs{\vt p} - \frac{\lambda^{\frac{1}{2}}(s_{12}, m_1^2, m_2^2)}{2 \sqrt{s_{12}}} \right) \dd \Omega \\
   & = \frac{g_1 g_2}{16 \pi^2}
  \frac{\lambda^{\frac{1}{2}}(s_{12}, m_1^2, m_2^2) / 2 \sqrt{s_{12}}}{E_1 + E_2} \dd \Omega                        \\
   & = g_1 g_2 \frac{\lambda^{\frac{1}{2}}(s_{12}, m_1^2, m_2^2)}{32 \pi^2 s_{12}} \dd \Omega                       \\
   & \stackrel{\int \dd \Omega}{=} g_1 g_2 \frac{\lambda^{\frac{1}{2}}(s_{12}, m_1^2, m_2^2)}{8 \pi s_{12}}
\end{align}

\subsubsection{Phase Space Decomposition}%
\label{sec:phase_space_decomposition}

We will be introducing new intermediate momenta \(q_{ij} = p_i + p_j\) such that
\(q_{ij}^2 = s_{ij}\), and then integrating over them and introducing Dirac
deltas in order to enforce momentum conservation.  We will use the following two
identities:
\begin{align}
  1 & = \int \ddbar^4{q_{ij}} (2 \pi)^4 \delta(q_{ij} - p_i - p_j) \Theta(q_{ij}^0) \\
  1 & = \int \ddbar{s_{ij}} (2\pi) \delta(s_{ij} - q_{ij}^2)
\end{align}
The first enforces momentum conservation, the second enforces an `on-shell'
condition.  The \(\Theta\) is the Heaviside theta function which is equal to 0
when its argument is negative, and 1 otherwise and ensures that the integration
happens for \(\sqrt{s_{ij}} > 0\) only.  Combined, they can be expressed as
\begin{equation}
  1 = \int \ddbar^3{\vt q_{ij}} \frac{1}{2 E_{ij}} \ddbar{s_{ij}} (2 \pi)^4 \delta^4(q_{ij} - p_i - p_j) = \int \dd \Pi_{ij} \ddbar{s_{ij}} (2 \pi)^4 \delta^4(q_{ij} - p_i - p_j)
\end{equation}
where \(E_{ij} = q_{ij}^0\) is the time-like component of \(q_{ij}\).

The goal now will be to express \(\dd \Phi_n(q; p_1, \dots, p_n)\) in terms of a
product of smaller phase spaces (and then recursively apply this until only
two-body phase spaces remain).

\begin{align}
  \dd \Phi_n(q; p_1, p_2, \dots, p_n)
   & = \dd \Pi_1 \dd \Pi_2 \cdots \dd \Pi_n (2 \pi)^4 \delta^4(p_1 + p_2 + \cdots + p_n - q)                 \\
   & = \dd \Pi_1 \dd \Pi_2 \cdots \dd \Pi_n (2 \pi)^4 \delta^4(p_1 + p_2 + \cdots + p_n - q)                 \\
   & \quad \times \dd \Pi_{12} \ddbar{s_{ij}} (2 \pi)^4 \delta^4(q_{12} - p_1 - p_2)
  \intertext{The combination of \(\dd \Pi_1 \dd \Pi_2 (2 \pi)^4 \delta^4(q_{12} - p_1 - p_2)\) is exactly equal to \(\dd \Phi_2(q_{12}; p_1, p_2)\).}
   & = \dd \Phi_2(q_{12}; p_1, p_2) \ddbar{s_{12}}                                                           \\
   & \quad \times \dd \Pi_{12} \dd \Pi_3 \cdots \dd \Pi_n (2\pi)^4 \delta^4(q_{12} + p_3 + \cdots + p_n - q) \\
   & = \dd \Phi_2(q_{12}; p_1, p_2) \ddbar{s_{12}} \dd \Phi_{n-1}(q; q_{12}, p_3, \ldots, p_n)
\end{align}

\subsection{\(1 \leftrightarrow n\) Interactions}

The reaction density is given by
\begin{equation}
  \gamma(a \to \vt b) \int \dd \Pi_a \dd \Pi_{\vt b} (2\pi)^4 \delta^4(p_a - p_{\vt c}) \abs{\pzcM(a \to \vt b)}^2 f_a^{(0)}
\end{equation}

% \subsubsection{\(1 \leftrightarrow 1\) Mixing}%
% \label{sec:1-to-1_mixing}

% The reaction density is given by
% \begin{equation}
%   \gamma(a \to b) = \int \dd \Pi_a \dd \Pi_b (2 \pi)^4 \delta^4(p_a - p_b) \abs{\pzcM(a \leftrightarrow b)}^2 f_a^{(0)}
% \end{equation}
% which can be evaluated using the 2-body phase space, with \(q_{ab} = 2 p_b\):
% \begin{align}
%   \dd \Pi_a \dd \Pi_b (2 \pi)^4 \delta^4(p_a - p_b)
%    & = \dd \Phi(q_{ab}; p_a, p_b)                                                                 \\
%    & = \frac{g_a g_b \lambda^{\frac{1}{2}}(q_{ab}^2, m_a^2, m_b^2)}{32 \pi^2 q_{ab}^2} \dd \Omega
% \end{align}
% Thus the reaction density is
% \begin{equation}
%   \gamma(a \to b) = \frac{4 \pi g_a g_b \lambda^{\frac{1}{2}(q_{ab}^2, m_a^2, m_b^2)}}{32 \pi^2 }\abs{\pzcM}^2 f_a^{(0)}
% \end{equation}

% The reverse rate is
% \begin{align}
%   \gamma(b \to a)
%    & = \int \dd \Pi_b \dd \Pi_a (2 \pi)^4 \delta^4(p_b - p_a) \abs{\pzcM}^2 f_b^{(0)}                      \\
%    & = \frac{4\pi g_a g_b \lambda^{\frac{1}{2}}(q_{ab}^2, m_a^2, m_b^2)}{32 \pi^2} \abs{\pzcM}^2 f_b^{(0)}
% \end{align}
% Thus the change in \(n_a\) is:
% \begin{align}
%   \ddfrac{n_a}{t}
%    & = \frac{n_b}{n_b^{(0)}} \gamma(b \to a) - \frac{n_a}{n_a^{(0)}} \gamma(a \to b)                                                                 \\
%    & = \frac{g_a g_b \lambda^{\frac{1}{2}}(q_{ab}^2, m_a^2, m_b^2)}{8\pi} \left[ \frac{n_b}{n_b^{(0)}} f_b^{(0)} - \frac{n_a}{n_a^{(0)}} f_a \right]
% \end{align}

\subsubsection{\(1 \leftrightarrow 2\) Decay}

The reaction density is given by
\begin{equation}
  \gamma(a \rightarrow bc) =
  \int \dd \Pi_a \dd \Pi_b \dd \Pi_c (2\pi)^4
  \delta^4(p_a - p_b - p_c)
  \abs{\pzcM(a \leftrightarrow bc)}^2 f_a^{(0)}.
\end{equation}
As we are summing over all internal degrees of freedom, the final amplitude has
no dependence on any integration variable and can be taken outside the integral.

The integration over the final state particles can be done easily as it is
exactly a two-particle phase space,
\begin{equation}
  \begin{aligned}
    \int \dd \Pi_b \dd \Pi_c (2\pi)^4 \delta^4(p_a - p_b - p_c)
     & \equiv \int \dd \Phi_2(p_a; p_b, p_c)                                    \\
    %  & = g_b g_c \frac{\lambda^{\frac{1}{2}}(m_a^2, m_b^2, m_c^2)}{32 \pi^2 m_a^2} \dd \Omega_{bc} \\
     & = g_b g_c \frac{\lambda^{\frac{1}{2}}(m_a^2, m_b^2, m_c^2)}{8 \pi m_a^2}
  \end{aligned}
\end{equation}
In the limit that \(m_b, m_c \to 0\), this simplifies to \(g_b g_c / 8 \pi\).

% When combined with the squared amplitude, can be related to the
% 0-temperature decay width as
% \begin{equation}
%   \begin{aligned}
%     \dd \Gamma
%            & = \frac{\abs{\pzcM}^2}{32\pi^2}
%     \overbrace{\frac{\lambda^{\frac{1}{2}}(m_a^2, m_b^2, m_c^2)}{2 m_a}}^{\abs{\vt p_a}}
%     \frac{1}{m_a^2} \dd \Omega_{bc}                                                                 \\
%     \Gamma & = \frac{\abs{\pzcM}^2}{16\pi} \frac{\lambda^{\frac{1}{2}}(m_a^2, m_b^2, m_c^2)}{m_a^3}
%   \end{aligned}
% \end{equation}

As for the integration over the initial particle's phase space:
\begin{equation}
  % \begin{aligned}
  \int \dd \Pi_a f_a^{(0)}
  = g_a \int f_a^{(0)} \frac{\abs{\vt p_a}}{2} \frac{\dd E_a \dd \Omega_a}{(2\pi)^3}
  = g_a \frac{m_a K_1(m_a \beta)}{4 \pi^2 \beta}
  % \end{aligned}
\end{equation}

This the interaction density can be expressed in a succinct closed form as
\begin{equation}
  \gamma(a \to bc) = \frac{g_a g_b g_c}{32 \pi^3} \abs{\pzcM}^2
  \lambda^{\frac{1}{2}}(m_a^2, m_b^2, m_c^2)
  \frac{K_1(m_a \beta)}{m_a \beta} .
\end{equation}
which goes towards 0 quite quickly when \(m_a \beta > 1\).  Having said that,
the forward interaction rate is scaled by \(n_a / n_a^{(0)}\) where \(n_a^{(0)}
\propto K_2(m_a \beta)\) which also goes to 0 quickly when \(m_a \beta > 1\).
Thus in order to avoid having \(0 / 0\) in the computation, we combine the two:
\begin{equation}
  \begin{aligned}
    \frac{N_a}{N_a^{(0)}} \gamma(a \to bc)
     & = N_a \left( \frac{g_a m_a^2 K_2(m_a \beta)}{2 \pi^2 \beta} \right)^{-1} \gamma(a \to bc) \\
     & = N_a g_b g_c \abs{\pzcM}^2
    \frac{\lambda^{\frac{1}{2}}(m_a^2, m_b^2, m_c^2)}{16 \pi m_a^3}
    \frac{K_1(m_a \beta)}{K_2(m_a \beta)}                                                        \\
     & = n_a \frac{\zeta(3)}{16 \pi} g_b g_c \abs{\pzcM}^2
    \lambda^{\frac{1}{2}}(m_a^2, m_b^2, m_c^2)
    \frac{1}{m_a^3 \beta^3}
    \frac{K_1(m_a \beta)}{K_2(m_a \beta)},
  \end{aligned}
\end{equation}
where \(N_a\) is the non-normalized number density, and \(n_a\) is normalized to
a single massless bosonic degree of freedom.  The ratio of Bessel function
behaves as
\begin{equation}
  \frac{K_1(x)}{K_2(x)} = \begin{cases}
    \frac{x}{2} + \calO(x^3)                                              & x \ll 1 \\
    1 - \frac{3}{2x} + \frac{15}{8 x^2} + \calO\left(\frac{1}{x^3}\right) & x \gg 1
  \end{cases}.
\end{equation}

The backward rate is given by
\begin{equation}
  \gamma(bc \to a) = \int \dd\Pi_a \dd\Pi_b \dd\Pi_c (2\pi)^4 \delta^4(p_a - p_b - p_c) \abs{\pzcM(a \leftrightarrow bc)}^2 f_b^{(0)} f_c^{(0)}
\end{equation}
The time-like component of the Dirac delta enforces that \(E_a = E_b + E_c\) and
thus we have that
\begin{equation}
  f_b^{(0)} f_c^{(0)} = e^{(E_b + E_c) \beta} = e^{E_a \beta} = f_a^{(0)}
\end{equation}
thus recovering the same equation the forward rate.  As for the rate scaled by
the number densities of the initial state particles, there is no easy
cancelation (unless the particle have \emph{exactly} the same masses).

\subsubsection{\(1 \leftrightarrow 3\) Decay}%
\label{sec:1-to-3_decay}

\subsection{\(2 \leftrightarrow n\) Scattering}%
\label{sec:2-to-n_scattering}

The reaction density is given by
\begin{equation}
  \gamma(ab \to \vt c)
  = \int \dd \Pi_a \dd \Pi_b \dd \Pi_{\vt c} (2 \pi)^4 \delta^4 (p_a + p_b - p_{\vt c}) \abs{\pzcM(ab \to \vt c)}^2 f_a^{(0)} f_b^{(0)}.
\end{equation}
The integration over the initial states can be done first
\begin{align}
  \gamma(ab \to \vt c)
   & = \int \dd \Pi_a \dd \Pi_b \dd \Pi_{\vt c} (2 \pi)^4 \delta^4 (p_a + p_b - p_{\vt c}) \abs{\pzcM(ab \to \vt c)}^2 f_a^{(0)} f_b^{(0)}                                                            \\
   & = \int \dd \Phi_2(q_{ab}; p_a, p_b) \ddbar{s} \dd \Pi_{ab} \dd \Pi_{\vt c} (2 \pi)^4 \delta^4(q_{ab} - p_{\vt c}) \abs{\pzcM(ab \to \vt c)}^2 f^{(0)}_a f^{(0)}_b
  \intertext{The zero-chemical-potential phase space distributions can be expressed in terms of the centre-of-mass energy: \(f^{(0)}_a f^{(0)}_b = \exp[-(E_a + E_b)\beta] = \exp[- E_{ab} \beta]\).  Furthermore, we can choose a reference frame such that the solid angle integration in \(\dd \Phi_2(q_{ab}; p_a, p_b)\) is trivial (and therefore \(\Phi_2(q_{ab}; p_a, p_b) = g_a g_b \lambda^{\frac{1}{2}}(s, m_a^2, m_b^2) / 8 \pi s\))}
   & = \int \ddbar{s} \underbrace{\Phi_2(q_{ab}; p_a, p_b) \dd \Pi_{\vt c} (2 \pi)^4 \delta^4(q_{ab} - p_{\vt c}) \abs{\pzcM(ab \to \vt c)}^2}_{\defeq \hat \sigma(s)} \dd \Pi_{ab} e^{-E_{ab} \beta} \\
   & = \int \ddbar{s} \hat \sigma(s) \ddbar^3{\vt p_{ab}} \frac{1}{2 E_{ab}} e^{-E_{ab} \beta}                                                                                                        \\
   & = \frac{1}{32 \pi^4} \int \hat \sigma(s) \abs{\vt p_{ab}} e^{-E_{ab} \beta} \dd E_{ab} \dd \Omega_{ab} \dd s                                                                                     \\
   & = \frac{1}{32 \pi^4} \int \hat \sigma(s) \sqrt{E_{ab}^2 - s} e^{-E_{ab} \beta} \dd E_{ab} \dd \Omega_{ab} \dd s                                                                                  \\
   & = \frac{1}{8 \pi^3 \beta} \int \hat \sigma(s) \sqrt{s} K_1(\sqrt{s} \beta) \dd s
\end{align}
Note that there is an alternative definition for \(\hat \sigma\) which has an
additional factor of \(8 \pi\), and thus changing the \(1 / 8 \pi^3\)
prefactor to \(1 / 64 \pi^4\).

\subsubsection{\(2 \leftrightarrow 2\) Scattering}%
\label{sec:2-to-2_scattering}

When both initial and final states have two particles, both initial and final
two-body phase spaces can be greatly simplified.

\begin{align}
  \hat \sigma(s)
   & = \Phi_2(q_{ab}; p_a, p_b) \int \dd \Phi_2(q_{ab}; p_c, p_d) \abs{\pzcM(ab \to cd)}^2                                                      \\
   & = \Phi_2(q_{ab}; p_a, p_b) \frac{g_c g_d \lambda^{\frac{1}{2}}(s, m_c^2, m_d^2)}{32 \pi^2 s} \int \dd \Omega_{cd} \abs{\pzcM(ab \to cd)}^2
  \intertext{The integration over \(\phi_{cd}\) can be done trivially as we are averaging over polarization.  One only needs only to integrate over \(\cos \theta_{cd}\).  We can arbitrarily orient this such \(\cos \theta_{cd} = \cos \theta_c\), and by using the relation \cref{eq:t_angular}, the angular integral can be related to an integral over the \(t\) Mandelstam variable:}
   & = \frac{g_a g_b \lambda^{\frac{1}{2}}(s, m_a^2, m_b^2)}{8 \pi s} \frac{g_c g_d \lambda^{\frac{1}{2}}(s, m_c^2, m_d^2)}{16 \pi s}
  \int \ddfrac{\cos \theta}{t} \dd t \abs{\pzcM(ab \to cd)}^2                                                                                   \\
   & = \frac{g_a g_b g_c g_d}{64 \pi^2 s} \int \dd t \abs{\pzcM(ab \to cd)}^2
\end{align}

As a result, the full expression for a \(2 \to 2\) reaction is:
\begin{align}
  \gamma(ab \to cd) = \frac{g_a g_b g_c g_d}{512 \pi^5 \beta} \int \abs{\pzcM(ab \to cd)}^2 \frac{K_1(\sqrt{s} \beta)}{\sqrt s} \dd s \dd t
\end{align}

\subsubsection{Mandelstam Variables}%
\label{sec:mandelstam_variables}

Note that in the centre-of-mass frame of reference, we have:
\begin{align}
  p_a & = (E_a, \abs{\vt p}, 0, 0)                                       & p_b & = (E_b, -\abs{\vt p}, 0, 0) \\
  p_c & = (E_c, \abs{\vt p} \cos \theta, 0, \abs{\vt p} \sin \theta)     &
  p_d & = (E_c, - \abs{\vt p} \cos \theta, 0, - \abs{\vt p} \sin \theta)
\end{align}
and the Mandelstam variables
\begin{align}
  s & = (p_a + p_b)^2 = (p_c + p_d)^2 \\
  t & = (p_a - p_c)^2 = (p_b - p_d)^2 \\
  u & = (p_a - p_d)^2 = (p_b - p_c)^2
\end{align}
By requiring everything to be on-shell, we can solve all the energies and
\(\abs{\vt p}\) in terms of only masses and \(s\):
\begin{align}
  E_a & = \frac{s + m_a^2 - m_b^2}{2 \sqrt{s}} &
  E_b & = \frac{s + m_b^2 - m_a^2}{2 \sqrt{s}}   \\
  E_c & = \frac{s + m_c^2 - m_d^2}{2 \sqrt{s}} &
  E_d & = \frac{s + m_d^2 - m_c^2}{2 \sqrt{s}}   \\
\end{align}
\begin{equation}
  \begin{aligned}
    \abs{\vt p}   & = \frac{\lambda^{\frac{1}{2}}(s, m_a^2, m_b^2)}{2 \sqrt{s}} = \frac{\lambda^{\frac{1}{2}}(s, m_c^2, m_d^2)}{2 \sqrt{s}} \\
    \abs{\vt p}^2 & = \frac{\lambda^{\frac{1}{2}}(s, m_a^2, m_b^2) \lambda^{\frac{1}{2}}(s, m_c^2, m_d^2)}{4 s}
  \end{aligned}
\end{equation}

Now the \(t\) Mandelstam variable is:
\begin{align}
  t & = (E_a - E_c)^2 - 2 \abs{\vt p}^2 + 2 \abs{\vt p}^2 \cos \theta                                                \\
    & = \frac{\left( m_a^2 - m_b^2 - m_c^2 + m_d^2 \right)^2}{4 s}
  - \frac{\lambda^{\frac{1}{2}}(s, m_a^2, m_b^2) \lambda^{\frac{1}{2}}(s, m_3^2, m_4^2)}{2 s}                        \\
    & \qquad + \frac{\lambda^{\frac{1}{2}}(s, m_a^2, m_b^2) \lambda^{\frac{1}{2}}(s, m_3^2, m_4^2)}{2 s} \cos \theta
\end{align}
And in particular, one can relate a differential in \(\cos \theta\) to a differential in \(t\):
\begin{equation}
  \label{eq:t_angular}
  \ddfrac{t}{\cos \theta} = 2 \abs{\vt p}^2 = \frac{\lambda^{\frac{1}{2}}(s, m_a^2, m_b^2) \lambda^{\frac{1}{2}}(s, m_c^2, m_d^2)}{2 s}
\end{equation}

As for the domain of integration of \(t\), it is not difficult to see that
\(\theta \in [0, \pi]\) corresponds to
\begin{equation}
  t \in \left[
    \frac{\left( m_a^2 - m_b^2 - m_c^2 + m_d^2 \right)^2}{4 s},
    \frac{\left( m_a^2 - m_b^2 - m_c^2 + m_d^2 \right)^2}{4 s}
    - \frac{\lambda^{\frac{1}{2}}(s, m_a^2, m_b^2) \lambda^{\frac{1}{2}}(s, m_c^2, m_d^2)}{s}
    \right]
\end{equation}

\section{Boltzmann Equations}%
\label{sec:boltzmann_equation}

We consider an interaction \(ab \leftrightarrow cd\), by \(\mathcal{CPT}\)
invariance we have that:
\begin{equation}
  \gamma(ab \to cd) = \gamma(\overline{c} \overline{d} \to \overline{a} \overline{b})
\end{equation}
and the \(\mathcal{CP}\) asymmetry is taken into account with \(\epsilon\) such
that
\begin{equation}
  \gamma(ab \to cd) - \gamma(\overline{a} \overline{b} \to \overline{c} \overline{d})
  = \epsilon
  = \gamma(\overline{c} \overline{d} \to \overline{a} \overline{b}) - \gamma(cd \to ab).
\end{equation}
and using \(\mathcal{CPT}\) invariance as well, this can be written as
\begin{equation}
  \epsilon = \gamma(ab \to cd) - \gamma(cd \to ab).
\end{equation}
As a result, any forward or backward process and their \(\mathcal{CP}\)
conjugate can expressed in terms of the original forward rate \(\gamma\), and
the asymmetry \(\epsilon\).

We will also be defined the asymmetry between the particle and anti-particle
number densities
\begin{equation}
  \Delta_a \defeq n_a - n_{\overline a}.
\end{equation}
and it is worth noting that generally
\begin{equation}
  \abs{\epsilon} \ll \gamma \qquad\qquad \abs{\Delta_a} \ll n_a,
\end{equation}
and in the \(\mathcal{CP}\)-conserving limit, both \(\epsilon\) and \(\Delta_a\)
go to zero.

The change in number density for \(a\) and \(\overline{a}\) are:
\begin{align}
  \ddfrac{n_a}{t}
   & = \frac{n_c n_d}{n_c^{(0)} n_d^{(0)}} \gamma(cd \to ab)
  - \frac{n_a n_b}{n_a^{(0)} n_b^{(0)}} \gamma(ab \to cd)                                                                      \\
   & = \left[
  \frac{n_c n_d}{n_c^{(0)} n_d^{(0)}}
  - \frac{n_a n_b}{n_a^{(0)} n_b^{(0)}}
  \right] \gamma
  + \frac{n_c n_d}{n_c^{(0)} n_d^{(0)}} \epsilon,                                                                              \\
  \ddfrac{n_{\overline a}}{t}
   & = \frac{n_{\overline c} n_{\overline d}}{n_c^{(0)} n_d^{(0)}} \gamma(\overline c \overline d \to \overline a \overline b)
  - \frac{n_{\overline a} n_{\overline b}}{n_a^{(0)} n_b^{(0)}} \gamma(\overline a \overline b \to \overline d \overline c)    \\
   & = \left[
  \frac{n_{\overline c} n_{\overline d}}{n_c^{(0)} n_d^{(0)}}
  - \frac{n_{\overline a} n_{\overline b}}{n_a^{(0)} n_b^{(0)}}
  \right] \gamma
  + \frac{n_{\overline a} n_{\overline b}}{n_a^{(0)} n_b^{(0)}} \epsilon.                                                      \\
\end{align}

In the \(\mathcal{CP}\) conserving limit, both of these simplify to
\begin{equation}
  \ddfrac{n_a}{t} = \ddfrac{n_{\overline a}}{t}
  = \left[ \frac{n_c n_d}{n_c^{(0)} n_d^{(0)}}  - \frac{n_a n_b}{n_a^{(0)} n_b^{(0)}} \right] \gamma(ab \to cd)
\end{equation}

Computing the rate of change of the \(\mathcal{CP}\) asymmetry (obviously before
taking any limits), we obtain:
\begin{equation}
  \begin{aligned}
    \ddfrac{\Delta_a}{t}
     & = \ddfrac{n_a}{t} - \ddfrac{n_{\overline a}}{t} \\
     & = \left[
    \frac{\Delta_c n_d + \Delta_d n_c}{n_c^{(0)} n_d^{(0)}}
    - \frac{\Delta_a n_b + \Delta_b n_a}{n_a^{(0)} n_b^{(0)}} \right] \gamma
    - \left[
    \frac{n_a n_b}{n_a^{(0)} n_b^{(0)}} + \frac{n_c n_d}{n_c^{(0)} n_d^{(0)}}
    \right] \epsilon
  \end{aligned}
\end{equation}
where we have omitted terms proportional to \(\Delta^2\) and \(\epsilon \Delta\)
(for any combination of \(\Delta_i\)).

\subsection{Three-Body Interactions}%
\label{sec:three-body_interactions}

Given a three-body interaction with (tree-level) squared amplitude
\(\abs{\pzcM}^2\), then there are six possible interactions (ignoring their
\(\mathcal{CPT}\) counterparts):
\begin{equation}
  \begin{aligned}
    a           & \to bc                         & \qquad
    \overline b & \to \overline ac               & \qquad
    \overline c \to \overline ab                          \\
    a           & \gets bc                       &
    \overline b & \gets \overline ac
                & \overline c \gets \overline ab
  \end{aligned}
\end{equation}
The number density for \(a\) therefore is:
\begin{equation}
  \begin{aligned}
    \ddfrac{n_a}{t} =
     & - \left( \frac{n_a}{n_a^{(0)}} - \frac{n_b}{n_b^{(0)}} \frac{n_c}{n_c^{(0)}} \right) \gamma(a \leftrightarrow bc)                       \\
     & + \left( \frac{n_b}{n_b^{(0)}} - \frac{n_a}{n_a^{(0)}} \frac{n_c}{n_c^{(0)}} \right) \gamma(\overline b \leftrightarrow \overline a c)  \\
     & + \left( \frac{n_c}{n_c^{(0)}} - \frac{n_a}{n_a^{(0)}} \frac{n_b}{n_b^{(0)}} \right) \gamma(\overline c \leftrightarrow \overline a b).
  \end{aligned}
\end{equation}

\subsection{Asymmetry Evolution}%
\label{sec:asymmetry_evolution}

We define the asymmetry in the phase space and number densities as
\begin{align}
  \Delta f_a & \defeq f_a - f_{\overline a}, &
  \Delta n_a & \defeq n_a - n_{\overline a},
\end{align}
and taking \(\mu_a = - \mu_{\overline a}\) (which is valid provided
interactions \(a \overline a \leftrightarrow (\dots)\) are fast), then we have
that
\begin{align}
  \Delta f_a & \defeq f_a^{(0)} \left( e^{-\mu \beta} - e^{\mu \beta} \right), &
  \Delta n_a & \defeq n_a^{(0)} \left( e^{-\mu \beta} - e^{\mu \beta} \right).
\end{align}
As a result, the asymmetry in the phase space can the related to the number
density asymmetry through
\begin{equation}
  \Delta f_a = f_a^{(0)} \left( e^{-\mu \beta} - e^{\mu \beta} \right)
  = f_a^{(0)} \frac{\Delta n_a}{n_a^{(0)}}
\end{equation}

The rate of change in the number density asymmetry is
\begin{equation}
  \begin{aligned}
    \ddfrac{\Delta n_a}{t}
     & = - \int_{\vt a}^{\vt b} \abs{\pzcM}^2 \left[ \prod_{i \in \vt a} \Delta f_i - \prod_{i \in \vt b} \Delta f_i \right]                                                         \\
     & = - \int_{\vt a}^{\vt b} \abs{\pzcM}^2 \left[ \prod_{i \in \vt a} \frac{\Delta n_i}{n_i^{(0)}} f_i^{(0)} - \prod_{i \in \vt b} \frac{\Delta n_i}{n_i^{(0)}} f_i^{(0)} \right] \\
     & = - \left[ \gamma(\vt a \to \vt b) \prod_{i \in \vt a} \frac{\Delta n_i}{n_i^{(0)}} - \gamma(\vt b \to \vt a) \prod_{i \in \vt b} \frac{\Delta n_i}{n_i^{(0)}} \right]
  \end{aligned}
\end{equation}
On the other hand, one can also look at the change in the number density
asymmetries from looking at the change in the number densities of the particle
and antiparticle respectively:
\begin{equation}
  \begin{aligned}
    \ddfrac{\Delta n_a}{t}
     & = \ddfrac{n_a}{t} - \ddfrac{n_{\overline a}}{t}                                                                                                                                                                                  \\
     & = - \left[ \gamma(\vt a \to \vt b) \prod_{i \in \vt a} \frac{n_i}{n_i^{(0)}} - \gamma(\vt b \to \vt a) \prod_{i \in \vt b} \frac{n_i}{n_i^{(0)}} \right]                                                                         \\
     & \quad + \left[ \gamma(\overline{\vt a} \to \overline{\vt b}) \prod_{i \in \vt a} \frac{n_{\overline i}}{n_i^{(0)}} - \gamma(\overline{\vt b} \to \overline{\vt a}) \prod_{i \in \vt b} \frac{n_{\overline i}}{n_i^{(0)}} \right] \\
     & = - \gamma(\vt a \to \vt b) \left[ \prod_{i \in \vt a} \frac{n_i}{n_i^{(0)}} + \prod_{i \in \vt b} \frac{n_{\overline i}}{n_i^{(0)}} \right]
    - \gamma(\vt b \to \vt a) \left[ \prod_{i \in \vt a} \frac{n_{\overline i}}{n_i^{(0)}} + \prod_{i \in \vt b} \frac{n_i}{n_i^{(0)}} \right]
  \end{aligned}
\end{equation}
where we have used \(\mathcal{CPT}\) invariance to equal the rates \(\vt a \to
\vt b\) and \(\overline{\vt b} \to \overline{\vt a}\).  Further assuming
\(\mathcal{CP}\) symmetry, this can be simplified further to
\begin{equation}
  \ddfrac{\Delta n_a}{t} = - \gamma(\vt a \leftrightarrow \vt b) \left[ \prod_{i \in \vt a} \frac{n_i}{n_i^{(0)}} + \prod_{i \in \vt a} \frac{n_{\overline i}}{n_i^{(0)}} + \prod_{i \in \vt b} \frac{n_{\overline i}}{n_i^{(0)}} + \prod_{i \in \vt b} \frac{n_i}{n_i^{(0)}} \right]
\end{equation}


\end{document}
