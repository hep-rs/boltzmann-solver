%%%%%%%%%%%%%%%%%%%%%%%%%%%%%%%%%%%%%%%%%%%%%%%%%%%%%%%%%%%%%%%%%%%%%%%%%%%%%%%%
%% HEADER
%%%%%%%%%%%%%%%%%%%%%%%%%%%%%%%%%%%%%%%%%%%%%%%%%%%%%%%%%%%%%%%%%%%%%%%%%%%%%%%%

%% The [KOMA-script](https://www.ctan.org/pkg/koma-script) package provides
%% alternatives to replacements for the default classes (`scrartcl`, `scrreprt`,
%% `scrbook`, `scrlttr2`) and provides a lot of options to more easily customize the
%% layout of the page.
\documentclass[
  a4paper,             % The size of the layout
  11pt,                % The font size (10pt, 11pt, 12pt)
  oneside,             % Whether the document is one-sided or two-sided
  onecolumn,           % Whether to have one or two columns in the layout
  bibliography=totoc,  % Adjust how the bibliography appears in ToC
  final,               % As opposed to draft (which speeds up compilation)
]{scrartcl}
%% One of the important options in the list above is `draft`/`final`.  If draft
%% mode is enabled, LaTeX and most packages will disable features which take up
%% a lot of time in order to speed up compilation.  Draft mode also highlights
%% certain error in the output document.  On the other hand, final mode enables
%% all features which will increase (sometimes quite significantly) the
%% compilation time of the document.

%% For the vast majority of packages, the order in which they are loaded does
%% not matter.  There are a few exceptions though, mostly relating to the
%% various referencing packages (hyperef, cleveref, ...).  The following package
%% must be loaded first and provides some additional functionalities to other
%% packages.
\usepackage{etex}
\usepackage{ifluatex}  % Provides the command \ifluatex

%% Formatting
%%%%%%%%%%%%%%%%%%%%%%%%%%%%%%%%%%%%%%%%%%%%%%%%%%%%%%%%%%%%%%%%%%%%%%%%%%%%%%%%

\usepackage{geometry}   % Customize text width, page height, margins, etc.
\geometry{
  %% See the geometry package's documentation for all the options available
  textwidth=13cm,
}

\usepackage{multicol}   % {multicols}{n} environment
\usepackage{pdflscape}  % {landscape} environment
\usepackage{pageslts}   % Improved page numbering
\usepackage{enumitem}   % Easily customize lists

\usepackage{setspace}   % Line spacing
\singlespacing%
% \onehalfspacing%
% \doublespacing%

%% Change the formatting of titles and sections.
\setkomafont{part}{\normalfont\scshape\Huge}
\setkomafont{partnumber}{\normalfont\scshape\huge}
\setkomafont{section}{\normalfont\Huge}
\setkomafont{subsection}{\normalfont\huge}
\setkomafont{subsubsection}{\normalfont\Large}
\setkomafont{paragraph}{\normalfont\large\scshape}

%% Change the formatting of section entries in the table of contents
\setkomafont{sectionentry}{\scshape}

%% Modify the indentation at the start of each paragraph.
\setlength{\parindent}{1em}

%% Font
%%%%%%%%%%%%%%%%%%%%%%%%%%%%%%%%%%%%%%%%%%%%%%%%%%%%%%%%%%%%%%%%%%%%%%%%%%%%%%%%

\usepackage{microtype}   % Fine small typographical details
\usepackage{realscripts} % Use the font's sub- and superscripts

% \ifluatex%
  \usepackage{fontspec}  % Allows for any font to be specified

  % Change the main font
  \setmainfont{EB Garamond}[
    % Contextuals={Alternate},
    % Ligatures={Historic},
    Ligatures={Rare},
    Numbers=OldStyle,
    % CharacterVariant={1},    % historical s
    % CharacterVariant={3},    % historical j
    % CharacterVariant={6},    % guillemets
    % CharacterVariant={11},   % distinguish i and 1
    % CharacterVariant={21},   % a
    % CharacterVariant={27},   % g
    ItalicFeatures={
      CharacterVariant={4},  % &
      CharacterVariant={5:3},  % v
    }
  ]


  % Change the mono-space font
  \setmonofont{Fira Code}[
    Scale=MatchLowercase,
  ]
% \else
%   \usepackage{ebgaramond}  % A fall-back that isn't as nice as fontspec
% \fi

\usepackage[cmintegrals,varg]{newtxmath}  % Nice math font with Garamond

%% Languages
%%%%%%%%%%%%%%%%%%%%%%%%%%%%%%%%%%%%%%%%%%%%%%%%%%%%%%%%%%%%%%%%%%%%%%%%%%%%%%%%

\usepackage[UKenglish]{babel}           % Set up the language
\usepackage[style=british]{csquotes}    % Advanced handling of quotes
\usepackage[perpage,symbol*]{footmisc}  % Have symbols instead of numbers for footnotes

%% siunitx is an incredible useful package to display numbers and their
%% associated units.  It also offers extra commands to specify lists and ranges
%% of values with units.  Some commands which it provides include:
%% - \SI{quantity}{units}
%% - \SIRange{start}{end}{units}
%% - \SIList{a, b, ..., d}{units}
%% With the units being given as `\kilo\metre\per\year\per\pico\barn`.
\usepackage{siunitx}

%% Adjust the way units are displayed in ranges and units so that they are only
%% mentinoed once.
\sisetup{
  range-units=single,
  list-units=single,
}

%% Declare a few extra units
\DeclareSIUnit\year{yr}
\DeclareSIUnit\barn{b}
\DeclareSIUnit\fb{\femto\barn}
\DeclareSIUnit\pb{\pico\barn}

%% Graphics & Figure
%%%%%%%%%%%%%%%%%%%%%%%%%%%%%%%%%%%%%%%%%%%%%%%%%%%%%%%%%%%%%%%%%%%%%%%%%%%%%%%%

%% Graphics can be included in a LaTeX document with
%%
%% ```
%% \includegraphics[options]{image}
%% ```
%%
%% Usually, the extension for the image is not specified and LaTeX will
%% figure out the best one.  It is of course possible to add the extension to
%% specify a particular image.
%%
%% In order to ensure that the image is a particular width, you can use
%% `width=<distance>`.  This will also scale the height but keep the aspect
%% ratio (if you also specify the height, the aspect ratio will not be kept).
%% It is tempting to specify an exact distance such as `4cm`; but it is
%% generally better to have a relative distance such as `0.4\linewidth`.  This
%% way, the image will scale nicely if the context is changed.
\usepackage{graphicx}   % Handle graphics

%% Look for images in the './images/' sub-directory
\graphicspath{{./images/}}

\usepackage{xcolor}     % Define and use colours
\usepackage{subcaption} % Subfigures inside a figure

\usepackage{tikz}       % Powerful drawing language
\usepackage{pgfplots}   % Plotting with LaTeX
\usepackage[compat=1.1.0]{tikz-feynman}  % Package to draw Feynman diagram
\pgfplotsset{compat=1.14}

%% TikZ pictures and plots can significantly increase the time it takes to
%% produce the output.  The `external` TikZ library library defers the creation
%% of these figures to a sub-process which creates a separate PDF file which is
%% then simply imported into the main document.  To call the sub-process, you
%% have to execute the appropriate makefile.  If you are using LatexMk, you can
%% use the `.latexmkrc` to automatically do this for you.
%%
%% The following setup works on Linux, and should work on OS X too.
% \usetikzlibrary{external}
% \tikzexternalize[shell escape=-shell-escape, prefix=images/tikz/]
% \immediate\write18{mkdir -p images/tikz/}
% \tikzset{
%     external/mode=list and make,
%     external/system call={
%         lualatex \tikzexternalcheckshellescape -halt-on-error -interaction=batchmode -jobname="\image" "\texsource" || rm "\image.pdf"},
% }

%% Math Packages
%%%%%%%%%%%%%%%%%%%%%%%%%%%%%%%%%%%%%%%%%%%%%%%%%%%%%%%%%%%%%%%%%%%%%%%%%%%%%%%%

\usepackage{amsmath}     % The core math package
% \usepackage{amssymb}     % Defines additional math fonts and symbols
\usepackage{mathtools}   % Various extra maths functions
\usepackage{dsfont}      % A double stroke font (for double stroke 1)
% \usepackage{autonum}   % Only number referenced equations (must be loaded after cleverref)

%% Define a new command called \withnumber which forces the line to have number
%% (even if it is referenced elsewhere).
\newcommand\withnumber{\refstepcounter{equation}\tag{\theequation}}

%% Allows page breaks in math (1 = avoid if possible, 4 = whenever).  Page
%% breaks can be avoided at particular places by using \\* instead of \\.
\allowdisplaybreaks[2]

%% Tables
%%%%%%%%%%%%%%%%%%%%%%%%%%%%%%%%%%%%%%%%%%%%%%%%%%%%%%%%%%%%%%%%%%%%%%%%%%%%%%%%

\usepackage{tabularx}  % Improved tabular environment
\usepackage{longtable} % Multipage tables
\usepackage{array}     % New column types, including >x<
\usepackage{booktabs}  % Provides nicer horizontal lines
\usepackage{multirow}  % Allows cells to span multiple rows
\usepackage{longtable} % Allows for tables to span multiple pages

%% Define the maths version of clr columns.
\newcolumntype{C}{>{\(}c<{\)}}
\newcolumntype{L}{>{\(}l<{\)}}
\newcolumntype{R}{>{\(}r<{\)}}

%% When designing tables, it is better to avoid having too many lines.  In
%% general, the vertical alignments will suffice to guide the eye and the only
%% horizontal lines present should demarcate particular sections within the
%% table (e.g. the title from the body, and the body from the averages at the
%% bottom).  The `booktabs` package provides `\toprule`, `\midrule` and
%% `\bottomrule` for this purpose.

%% Code
%%%%%%%%%%%%%%%%%%%%%%%%%%%%%%%%%%%%%%%%%%%%%%%%%%%%%%%%%%%%%%%%%%%%%%%%%%%%%%%%

\usepackage{minted}

%% Linking and Referencing
%%%%%%%%%%%%%%%%%%%%%%%%%%%%%%%%%%%%%%%%%%%%%%%%%%%%%%%%%%%%%%%%%%%%%%%%%%%%%%%%

\usepackage{hyperref}  % Automatically inserts hyperlinks.
\usepackage{cleveref}  % Use `\cref` to reference anything
% \usepackage{autonum}   % Must be loaded after all *ref packages

%% Define some slightly nicer colors
\definecolor{link-color}{RGB}{96 0 0}
\definecolor{cite-color}{RGB}{0 96 0}
\definecolor{file-color}{RGB}{0 0 96}
\definecolor{url-color}{RGB}{0 0 96}
\definecolor{link-border-color}{RGB}{255 159 159}
\definecolor{cite-border-color}{RGB}{159 255 159}
\definecolor{file-border-color}{RGB}{159 159 255}
\definecolor{url-border-color}{RGB}{159 159 255}

\hypersetup{
  %% When `colorlinks` is true, all links will be coloured which looks nice in
  %% digital version of the document but not in print.  If the document is
  %% intended for printing, then `colorlinks` should set to false.
  colorlinks=true,
  linkcolor=link-color,
  citecolor=cite-color,
  filecolor=file-color,
  urlcolor=url-color,
  linkbordercolor=link-border-color,
  citebordercolor=cite-border-color,
  urlbordercolor=url-border-color,
}

%% Glossary
%%%%%%%%%%%%%%%%%%%%%%%%%%%%%%%%%%%%%%%%%%%%%%%%%%%%%%%%%%%%%%%%%%%%%%%%%%%%%%%%
%% This package requires `makeglossaries` to be run after the initial run of
%% LaTeX so that the glossary is generated and the a second run of LaTeX is need
%% to included the newly generated glossary.  (This is automatically handled by
%% Latexmk with the provided .latexmkrc).

%% hyperref should be loaded first
\usepackage[toc]{glossaries}
\usepackage{glossaries-extra}

\setglossarystyle{index}
\setabbreviationstyle[acronym]{long-short-sc}

\loadglsentries{glossary.glos}

\makeglossaries%

%% Bibliography
%%%%%%%%%%%%%%%%%%%%%%%%%%%%%%%%%%%%%%%%%%%%%%%%%%%%%%%%%%%%%%%%%%%%%%%%%%%%%%%%

%% hyperref should be loaded first
\usepackage[
  %% See the extensive documentation for biblatex for a description of what
  %% these options do
  autocite=inline,
  backend=biber,
  biblabel=brackets,
  doi=true,
  eprint=true,
  maxnames=4,
  style=phys,
]{biblatex}

%% Add a bib file
\addbibresource{references.bib}

%% Other modifications/package
%%%%%%%%%%%%%%%%%%%%%%%%%%%%%%%%%%%%%%%%%%%%%%%%%%%%%%%%%%%%%%%%%%%%%%%%%%%%%%%%

%% This is not a CTAN package but instead contains a whole lot of small scripts I
%% use nearly all the time.
\usepackage{jpellis}

%% Document Information
%%%%%%%%%%%%%%%%%%%%%%%%%%%%%%%%%%%%%%%%%%%%%%%%%%%%%%%%%%%%%%%%%%%%%%%%%%%%%%%%

%% Define a few shorthand commands.  The `makeatletter' and `makeatother' allows
%% the use of '@' in commands which is reserved for hidden functions.
\makeatletter
\newcommand\@degreetitle{}
\newcommand\degreetitle[1]{\renewcommand\@degreetitle{#1}}

\newcommand\@department{}
\newcommand\department[1]{\renewcommand\@department{#1}}

\newcommand\@university{}
\newcommand\university[1]{\renewcommand\@university{#1}}

\newcommand\@keywords{}
\newcommand\keywords[1]{\renewcommand\@keywords{#1}}

\AtEndPreamble{
  %% Set the PDF metadata, though this can only be done once they've been all
  %% defined which is at the end of the preamble.
  \hypersetup{
    pdftitle={\@title},
    pdfauthor={\@author},
    pdfsubject={\@subject},
    pdfkeywords={\@keywords},
  }
}
\makeatother

%% The \texorpdfstring string will, depending on context, either print proper
%% TeX command (in this case LaTeX will the special kerning), or the plain text
%% fallback.  This is useful if you have equation, links or special formatting
%% in titles as the PDF bookmarks can't handle them.
\title{Boltzmann Solver}
% \subtitle{A Practical Introduction to \texorpdfstring{\LaTeX}{LaTeX}}

\subject{Boltzmann Solver}
\keywords{boltzmann equation, numerical method, ordinary differential equation}

\author{Joshua P.~\textsc{Ellis}}
\degreetitle{Doctor of Philosphy (Physics)}
\department{School of Physics}
\university{The University of Melbourne}

\date{\today}

%%%%%%%%%%%%%%%%%%%%%%%%%%%%%%%%%%%%%%%%%%%%%%%%%%%%%%%%%%%%%%%%%%%%%%%%%%%%%%%%
%% DOCUMENT
%%%%%%%%%%%%%%%%%%%%%%%%%%%%%%%%%%%%%%%%%%%%%%%%%%%%%%%%%%%%%%%%%%%%%%%%%%%%%%%%
\begin{document}
%% Adjust the pagenumbering to capital Latin letters
\pagenumbering{Alph}

%% Title
%%%%%%%%%%%%%%%%%%%%%%%%%%%%%%%%%%%%%%%%%%%%%%%%%%%%%%%%%%%%%%%%%%%%%%%%%%%%%%%%
%% We don't want any page numbers, chapter headings, etc. on the page
\pagestyle{empty}

\begin{titlepage}
  \makeatletter
  \begin{center}
    \vspace*{2.5cm}

    {\Huge \@title} \\[0em]
    \rule{\linewidth}{2pt}
    {\huge \textsc{\@subtitle}} \\[6em]

    {\large By} \\[1cm]
    {\huge \@author} \\[0.5ex]
    {\Large \normalfont\@degreetitle}

    \vfill

    {\Large \@department} \\[1ex]
    {\Large \@university}

    \vfill

    {\large \@date}
  \end{center}

\makeatother
\end{titlepage}

%% Table of Content
%%%%%%%%%%%%%%%%%%%%%%%%%%%%%%%%%%%%%%%%%%%%%%%%%%%%%%%%%%%%%%%%%%%%%%%%%%%%%%%%
\cleardoublepage%
\pagestyle{plain}
\pagenumbering{roman}

\tableofcontents

%%%%%%%%%%%%%%%%%%%%%%%%%%%%%%%%%%%%%%%%%%%%%%%%%%%%%%%%%%%%%%%%%%%%%%%%%%%%%%%%
%% CONTENT
%%%%%%%%%%%%%%%%%%%%%%%%%%%%%%%%%%%%%%%%%%%%%%%%%%%%%%%%%%%%%%%%%%%%%%%%%%%%%%%%
\cleardoublepage%
\pagenumbering{arabic}

\section{Phase Space integration}%
\label{sec:phase_space_integration}

Given an \(n\) particle initial or final state, the phase space integration is
given by:
\begin{equation}
  \dd \Phi_n(q; p_1, \dots, p_n)
  \defeq \dd \Pi_1 \cdots \dd \Pi_n (2 \pi)^4
  \delta^4(p_1 + \cdots + p_n - q)
\end{equation}
where
\begin{equation}
  \dd \Pi
  \defeq \frac{g}{2 E} \ddbar^3{\vt p}
  = g \frac{\vt p^2}{2 E} \frac{\dd \abs{\vt p} \dd \Omega}{(2\pi)^3}
  = g \frac{\abs{\vt p}}{2} \frac{\dd E \dd \Omega}{(2 \pi)^3}
\end{equation}
where the last equality is achieved by noting that \(m^2 = E^2 - \abs{\vt p}^2\)
thereby implying that \(\abs{\vt p} \dd \abs{\vt p} \equiv E \dd E\).

\subsection{Two-Body Phase Space}%
\label{sec:two-body_phase_space}

The two-body phase space, \(\Phi_2\), is the smallest non-trivial space that can
be integrated over easily.  For convenience, we define
\begin{subequations}
\begin{align}
  q_{12} &\defeq p_1 + p_2, \\
  s_{12} &\defeq q_{12}^2 = (p_1 + p_2)^2,
\end{align}
\end{subequations}
and we can express a two-body phase space in relation to the sum of their
momenta:
\begin{subequations}
  \begin{align}
    \dd \Phi_2(q_{12}; p_1, p_2)
    & = \dd \Pi_1 \dd \Pi_2 (2 \pi)^4 \delta^4(p_1 + p_2 - q_{12})                                                   \\
    & = \frac{g_1}{2 E_1} \ddbar^3{\vt p_1} \frac{g_2}{2 E_2} \ddbar^3{\vt p_2}
    (2\pi) \delta(E_1 + E_2 - E_{12}) (2\pi)^3 \delta^3(\vt p_1 + \vt p_2 - \vt q_{12})                               \\
    %  & = \frac{g_1 g_2}{4 (2 \pi)^2} \frac{\dd^3 \vt p_1}{E_1} \frac{\dd^3 \vt p_2}{E_2}
    % \delta(E_1 + E_2 - E_{12}) \delta^3(\vt p_1 + \vt p_2 - \vt q_{12})                                               \\
    \intertext{The integral over \(\vt p_2\) can be done trivially with the Dirac delta (note that after this integration, \(E_2\) no longer depends on \(\abs{\vt p_2}\) and now depends on \(\abs{\vt q_{12} - \vt p_1}\))}
    & = \frac{g_1 g_2}{16 \pi^2} \frac{\dd^3 \vt p_1}{E_1} \frac{1}{E_2}
    \delta(E_1 + E_2 - E_{12})
    \intertext{The final quantity is Lorentz invariant, so we may choose the centre-of-mass frame for the calculations in which \(p_1 = (E_1, \vt p)\), \(p_2 = (E_2, - \vt p)\), and \(q_{12} = (\sqrt{s_{12}}, \vt 0)\).}
    & = \frac{g_1 g_2}{16 \pi^2} \frac{\abs{\vt p_1}^2 \dd \abs{\vt p_1}}{E_1 E_2}
    \delta(E_1 + E_2 - \sqrt{s_{12}}) \dd \Omega
    \intertext{Given that \(E_1 = \sqrt{\abs{\vt p_1}^2 + m_1^2}\) and similarly for \(E_2\), the zero of the Dirac delta can be found at \(\abs{\vt p_1} = \lambda^{\frac{1}{2}}(s_{12}, m_1^2, m_2^2) / 2 \sqrt{s_{12}}\):}
    & = \frac{g_1 g_2}{16 \pi^2} \frac{\abs{\vt p_1}^2 \dd \abs{\vt p_1}}{E_1 E_2}
    \left[ \ddfrac{(E_1 + E_2)}{\abs{\vt p_1}} \right]^{-1}
    \delta\left( \abs{\vt p_1} - \frac{\lambda^{\frac{1}{2}}(s_{12}, m_1^2, m_2^2)}{2 \sqrt{s_{12}}} \right) \dd \Omega \\
    & = \frac{g_1 g_2}{16 \pi^2} \dd \abs{\vt p_1} \frac{\abs{\vt p_1}^2 }{4 E_1 E_2}
    \left[ \frac{\abs{\vt p_1}}{E_1} + \frac{\abs{\vt p_1}}{E_2} \right]^{-1}
    \delta\left( \abs{\vt p_1} - \frac{\lambda^{\frac{1}{2}}(s_{12}, m_1^2, m_2^2)}{2 \sqrt{s_{12}}} \right) \dd \Omega \\
    & = \frac{g_1 g_2}{16 \pi^2}
    \frac{\lambda^{\frac{1}{2}}(s_{12}, m_1^2, m_2^2) / 2 \sqrt{s_{12}}}{E_1 + E_2} \dd \Omega                        \\
    & = \frac{g_1 g_2}{32 \pi^2} \frac{\lambda^{\frac{1}{2}}(s_{12}, m_1^2, m_2^2)}{s_{12}} \dd \Omega                       \\
    & \stackrel{\int \dd \Omega}{=} \frac{g_1 g_2}{8 \pi} \frac{\lambda^{\frac{1}{2}}(s_{12}, m_1^2, m_2^2)}{s_{12}}
  \end{align}
\end{subequations}

\subsection{Phase Space Decomposition}%
\label{sec:phase_space_decomposition}

We will be introducing new intermediate momenta \(q_{ij} = p_i + p_j\) and
variable \(s_{ij}\) and then integrating over them both and introducing Dirac
deltas in order to enforce momentum conservation.  We will use the following two
identities:
\begin{subequations}
  \begin{align}
    1 & = \int \ddbar^4{q_{ij}} (2 \pi)^4 \delta(q_{ij} - p_i - p_j) \\
    1 & = \int \dd {s_{ij}} \delta(s_{ij} - q_{ij}^2) \Theta(q_{ij}^0)
  \end{align}
\end{subequations}
The first enforces momentum conservation, the second enforces an `on-shell'
condition.  Combined, they can be expressed as
\begin{subequations}%
  \label{eq:phase_space_identity}
  \begin{align}
    1 &= \int \ddbar^4{q_{ij}} \dd {s_{ij}} (2 \pi)^4 \delta^4(q_{ij} - p_i - p_j) \delta(s_{ij} - q_{ij}^2) \Theta(q_{ij}^0) \\
    &= \int \ddbar^3{\vt q_{ij}} \ddbar{q_{ij}^0}  \dd {s_{ij}} (2 \pi)^4 \delta^4(q_{ij} - p_i - p_j) \delta(s_{ij} - q_{ij}^2) \Theta(q_{ij}^0)
    \intertext{
      The zeros of the last Dirac delta are located at \(q_{ij}^0 = \pm \sqrt{\vt q_{ij}^2 + s_{ij}}\)
      and the Heaviside theta function ensures the positive value is chosen, so we can perform the integration over \(q_{ij}^0\).
      As the Dirac delta depends on \((q_{ij}^0)^2\), a factor of \(1 / 2 q_{ij}^0 = 1 / 2 E_{ij}\) appears
      in the integrand and we obtain
    }
    &= \int \ddbar^3{\vt q_{ij}} \frac{1}{2 E_{ij}} \ddbar{s_{ij}} (2 \pi)^4 \delta^4(q_{ij} - p_i - p_j) \\
    &= \int \dd \Pi_{ij} \ddbar{s_{ij}} (2 \pi)^4 \delta^4(q_{ij} - p_i - p_j)
  \end{align}
\end{subequations}
where \(E_{ij} = q_{ij}^0\) is the time-like component of \(q_{ij}\).

The goal now will be to express \(\dd \Phi_n(q; p_1, \dots, p_n)\) in terms of a
product of smaller phase spaces (and then recursively apply this until only
two-body phase spaces remain).

\begin{subequations}
  \begin{align}
    \dd \Phi_n(q; p_1, p_2, \dots, p_n)
    & = \dd \Pi_1 \dd \Pi_2 \cdots \dd \Pi_n (2 \pi)^4 \delta^4(p_1 + p_2 + \cdots + p_n - q)                 \\
    & = \dd \Pi_1 \dd \Pi_2 \cdots \dd \Pi_n (2 \pi)^4 \delta^4(p_1 + p_2 + \cdots + p_n - q)                 \\
    & \quad \times \dd \Pi_{12} \ddbar{s_{ij}} (2 \pi)^4 \delta^4(q_{12} - p_1 - p_2)
    \intertext{The combination of \(\dd \Pi_1 \dd \Pi_2 (2 \pi)^4 \delta^4(q_{12} - p_1 - p_2)\) is exactly equal to \(\dd \Phi_2(q_{12}; p_1, p_2)\).}
    & = \dd \Phi_2(q_{12}; p_1, p_2) \ddbar{s_{12}}                                                           \\
    & \quad \times \dd \Pi_{12} \dd \Pi_3 \cdots \dd \Pi_n (2\pi)^4 \delta^4(q_{12} + p_3 + \cdots + p_n - q) \\
    & = \dd \Phi_2(q_{12}; p_1, p_2) \ddbar{s_{12}} \dd \Phi_{n-1}(q; q_{12}, p_3, \ldots, p_n)
  \end{align}
\end{subequations}

\clearpage
\section{Phase Space and Number Densities}

In thermal equilibrium, the phase space of a particle species is described by
\begin{equation}
  f(\beta; E, m, \mu) = \frac{g}{e^{(E - \mu) \beta} + \xi}
\end{equation}
where \(\xi\) is \(+1\), \(-1\) or \(0\) for Fermi--Dirac, Bose--Einstein and
Maxwell--Boltzmann statistics respectively.

The associated number density for this particle is
\begin{equation}
  N(\beta; m, \mu) = \int f \ddbar^3{\vt p} = \frac{1}{2 \pi^2} \int_m^\infty f E \sqrt{E^2 - m^2} \dd E.
\end{equation}
The integral can only be analytically evaluated for the Maxwell--Boltzmann
distribution:
\begin{equation}
  N_{\textsc{mb}}(\beta; m, \mu) = \frac{g}{2 \pi^2} \frac{m^2 K_2(m \beta)}{\beta} e^{\mu \beta}.
\end{equation}

The massless case can be evaluated for all three distributions:
\begin{subequations}
  \begin{align}
    N(\beta; 0, \mu) &= \frac{g}{\pi^2 \beta^3} \times \begin{cases}
       \Li_3(e^{\mu \beta}) & \text{Bose--Einstein} \\
       - \Li_3(- e^{\mu \beta}) & \text{Fermi--Dirac} \\
       e^{\mu \beta} & \text{Maxwell--Boltzmann}
    \end{cases} \\
    &= \frac{g}{\pi^2 \beta^3} \times \begin{cases}
       \zeta(3) + \frac{\pi^2}{6} \mu \beta + \calO(\mu^2) & \text{Bose--Einstein} \\
       \frac{3 \zeta(3)}{4} + \frac{\pi^2}{12} \mu \beta + \calO(\mu^2) & \text{Fermi--Dirac} \\
       1 + \mu \beta + \calO(\mu^2) & \text{Maxwell--Boltzmann}
    \end{cases}
  \end{align}
\end{subequations}
where \(\Li_3\) is the polylogarithm function, \(\zeta\) is the Riemann zeta
function and \(\zeta(3) \approx 1.20206\).

It is conventional to normalize the number density to a reference point that has
a constant comoving number density.  Two common conventions include the entropy
density and the photon number density.  In this work, we will use the
equilibrium number density of a single massless bosonic degree of freedom, which
is equivalent to half the photon density.

\begin{subequations}
  \begin{align}
    n_{\textsc{mb}}(\beta; m, \mu) &= \frac{g}{2 \zeta(3)} m^2 \beta^2 K_2(m \beta) e^{\mu \beta} \\
    n(\beta; 0, \mu) &= \frac{g}{\zeta(3)} \times \begin{cases}
       \Li_3(e^{\mu \beta}) & \text{Bose--Einstein} \\
       - \Li_3(- e^{\mu \beta}) & \text{Fermi--Dirac} \\
       e^{\mu \beta} & \text{Maxwell--Boltzmann}
    \end{cases} \\
    &= g \times \begin{cases}
       1 + \frac{\pi^2}{6 \zeta(3)} \mu \beta + \calO(\mu^2) & \text{Bose--Einstein} \\
       \frac{3}{4} + \frac{\pi^2}{12 \zeta(3)} \mu \beta + \calO(\mu^2) & \text{Fermi--Dirac} \\
       \frac{g}{\zeta(3)} + \frac{1}{\zeta(3)} \mu \beta + \calO(\mu^2) & \text{Maxwell--Boltzmann}
    \end{cases}
  \end{align}
\end{subequations}

One important property that will be used to make the interaction rate
calculations more tractable is to assume that the particles following a
Maxwell--Boltzmann distribution.  In that case, one important property that
emerges is the following relation:
\begin{equation}
  \label{eq:phase_space_to_number_density}
  \frac{f(\beta; E, m, \mu)}{f(\beta; E, m, 0)} = e^{\mu \beta} = \frac{N(\beta; m, \mu)}{N(\beta; m, 0)}
\end{equation}
which allows the phase space to be related to the equilibrium phase space and
the ratio of number densities.  This assumption is still approximately true for
the quantum statistical distributions provided the chemical potential is small.

\subsection{Number Density Asymmetry}%
\label{sec:number_density_asymmetry}

The number density asymmetry for a particle \(p\) is defined to be
\begin{equation}
  \Delta \defeq N - \overline N
\end{equation}
or the equivalent using the normalized number density.  The asymmetry can be
evaluated by computing the integral
\begin{equation}
  \Delta = \int f - \overline f \ddbar{\vt p} 
\end{equation}
and provided that the particle and antiparticle have a fast
co-annihilation/pair-creation interaction, the chemical potentials of both will
be equal and opposite.

As with the number density, this can only be analytically evaluated for the
Maxwell--Boltzmann distribution:
\begin{equation}
  \Delta_{\textsc{mb}}(\beta; m, \mu)
  = \frac{g}{2 \pi^2} \frac{m^2 K_2(m \beta)}{\beta} \sinh(\mu \beta)
\end{equation}
and the massless case can be evaluated for all three distributions:
\begin{subequations}
  \begin{align}
    \Delta(\beta; 0, \mu) &= \frac{g}{\pi^2 \beta^3} \times \begin{cases}
      \frac{1}{12} \left[ 2 \pi^2 \mu\beta - (\mu\beta)^3 \right] & \text{Bose--Einstein} \\
      \frac{1}{12} \left[ \pi^2 \mu\beta + (\mu\beta)^3 \right] & \text{Fermi--Dirac} \\
      \sinh(\mu \beta) & \text{Maxwell--Boltzmann}
    \end{cases}.
  \end{align}
\end{subequations}
For normalized number density asymmetries, the above evaluate to
\begin{equation}
  \Delta_{\textsc{mb}}(\beta; m, \mu)
  = \frac{g}{2 \zeta(3)} \frac{\beta^2 m^2 K_2(m \beta)}{\beta} \sinh(\mu \beta)
\end{equation}
and the massless case can be evaluated for all three distributions:
\begin{subequations}
  \begin{align}
    \Delta(\beta; 0, \mu) &= \frac{g}{\zeta(3)} \times \begin{cases}
      \frac{1}{12} \left[ 2 \pi^2 \mu\beta - (\mu\beta)^3 \right] & \text{Bose--Einstein} \\
      \frac{1}{12} \left[ \pi^2 \mu\beta + (\mu\beta)^3 \right] & \text{Fermi--Dirac} \\
      \sinh(\mu \beta) & \text{Maxwell--Boltzmann}
    \end{cases}
  \end{align}
\end{subequations}

For small chemical potentials, the phase space differences is
\begin{equation}
  f - \overline f = \frac{2 \beta \mu e^{E \beta}}{(e^{E \beta} + \xi)^2} + \calO(\mu^3)
\end{equation}
and therefore an alternative expression for the asymmetry is
\begin{equation}
  \Delta = 2 \beta \mu \int \frac{e^{E \beta}}{(e^{E \beta} + \xi)^2} \ddbar^3{\vt p}
\end{equation}
which unfortunately does not have a closed analytic form save for \(\xi = 0\).

\clearpage
\section{Boltzmann Equations}%
\label{sec:boltzmann_equations}

In full generality, the boltzmann equation for a particle with phase space
distribution \(f\) is
\begin{equation}
  \vt L[f] = \vt C[f]
\end{equation}
where \(\vt L\) is the Liouville operator and \(\vt C\) is the collision
operator.  The most general form of the phase space is a function of both
position, momentum and time; however, assuming the Universe to be homogeneous
and isotropic implies that the phase space only depends on energy and time.

\subsection{Liouville Operator}%
\label{sec:liouville_operator}

The Liouville operator in general is
\begin{equation}
  \vt L[f] = \left[ p^\mu \partial_\mu - \Gamma^\mu_{\nu\sigma} p^\nu p^\sigma \partial_\mu \right] f,
\end{equation}
under the \textsc{flrw} metric is
\begin{equation}
  \vt L[f] \stackrel{\textsc{flrw}}{=} \left[E \pfrac{}{t} - H \abs{\vt p}^2 \pfrac{}{E}\right] f
\end{equation}
where \(H \defeq \dot a / a\) is Hubble's constant.

As we are not interested in the phase space directly but instead the number
density of the particle species interacting, we integrate the Liouville operator
over the phase space of the particle (where \(g\) is the number of internal
degrees of freedom of the particle):
\begin{subequations}
  \begin{align}
    g \int \vt L[f] \dd \Pi
    &= \int \left[E \pfrac{}{t} - H \abs{\vt p}^2 \pfrac{}{E}\right] f \frac{g}{2E} \ddbar^3{\vt p} \\
    &= \frac{g}{2} \int \left[\pfrac{}{t} - H \frac{E^2 - m^2}{E} \pfrac{}{E}\right] f \ddbar^3{\vt p} \\
    &= g \int E \pfrac{f}{t} + 2 H E f - \pfrac{}{E} \left[H (E^2 - m^2) f \right] \ddbar^3{\vt p} \\
    &= g \int E \pfrac{f}{t} + 2 H E f \ddbar^3{\vt p} - g \left[H (E^2 - m^2) f \right]_{E = m}^{E = \infty} \\
    &= \pfrac{N}{t} - 3 H N
  \end{align}
\end{subequations}

\subsection{Collision Operator}%
\label{sec:collision_operator}

As for the collision operator, only the inelastic contributions survive.  Taking
the interaction to be \(\vt a \leftrightarrow \vt b\) for some set of particles
\(\vt a\) annihilating into \(\vt b\).  The collision term is:
\begin{equation}
  \begin{aligned}
    g_{a_1} \int \vt C[f_{a_1}] \dd \Pi_{a_1}
    &= \int_{\vt a}^{\vt b} \left[ \left(\prod_{i \in \vt a} f_i\right) \left(\prod_{i \in \vt b} 1 - \xi_i f_i\right) \abs{\pzcM_{\vt a}^{\vt b}}^2 \right. \\
    &\quad\qquad \left. - \left(\prod_{i \in \vt b} f_i\right) \left(\prod_{i \in \vt a} 1 - \xi_i f_i\right) \abs{\pzcM_{\vt b}^{\vt a}}^2 \right]
  \end{aligned}
\end{equation}
where we have introduced the following abbreviations:
\begin{subequations}
  \begin{align}
    \int_{\vt a}^{\vt b} &\defeq (2\pi)^4 \delta(p_{\vt a} - p_{\vt b}) \dd \Pi_{\vt a} \dd \Pi_{\vt b}, \\
    \abs{\pzcM_{\vt a}^{\vt b}}^2 &\defeq \abs{\pzcM(\vt a \to \vt b)}^2, \\
    p_{\vt a} &\defeq \sum_{i \in \vt a} p_i, \\
    \dd \Pi_{\vt a} &\defeq \prod_{i \in \vt a} \dd \Pi_i, \\
    \dd \Pi_i &\defeq \frac{g_i}{2 E_i} \ddbar^3{\vt p_i} \equiv \frac{g_i \abs{\vt p_i}}{2} \frac{\dd E_i \dd \Omega_i}{(2 \pi)^3}
  \end{align}
\end{subequations}
and the squared amplitude is \emph{summed} over all internal degrees of freedom.
Furthermore, the phase space is described by
\begin{equation}
  f_i \defeq \frac{1}{e^{(E_i - \mu_i) \beta} + \xi_i}
\end{equation}
where \(\xi_i\) is \(+1\) for fermions, \(-1\) for bosons and \(0\) if we wish
to use a Maxwell--Boltzmann distribution.  The quantum statistical factors are
\begin{equation}
  1 - \xi_i f_i \equiv e^{(E_i - \mu_i) \beta} f_i
\end{equation}

If we assume particles phase space can be reasonably approximated by a
Maxwell--Boltzmann distribution, then the quantum statistic factors can be
omitted and we can replace the phase space factors with equilibrium phase space
scaled by ratios of number densities (see
\cref{eq:phase_space_to_number_density}):
\begin{equation}
  \int_{\vt a}^{\vt b} \left[ \abs{\pzcM_{\vt a}^{\vt b}}^2 \left( \prod_{i \in \vt a} \frac{n_i}{n_i^{(0)}} f_i^{(0)} \right)
  - \abs{\pzcM_{\vt b}^{\vt a}}^2 \left( \prod_{i \in \vt b} \frac{n_i}{n_i^{(0)}} f_i^{(0)} \right) \right]
\end{equation}
Energy conservation ensures that \(\prod_{i \in \vt a} f_i^{(0)} = \prod_{i \in
\vt b} f_i^{(0)}\) and we may further simplify
\begin{equation}
  \int_{\vt a}^{\vt b} \prod_{i \in \vt a} f_i^{(0)} \left[
    \abs{\pzcM_{\vt a}^{\vt b}}^2 \left( \prod_{i \in \vt a} \frac{n_i}{n_i^{(0)}} \right)
    - \abs{\pzcM_{\vt b}^{\vt a}}^2 \left( \prod_{i \in \vt b} \frac{n_i}{n_i^{(0)}} \right)
  \right]
\end{equation}
Finally, if we define the CP asymmetry in the squared amplitude as
\begin{equation}
  \abs{\delta \pzcM_{\vt a}^{\vt b}}^2 \defeq \abs{\pzcM_{\vt a}^{\vt b}}^2 - \abs{\pzcM_{\vt b}^{\vt a}}^2
\end{equation}
then we obtain the final simplification for the interaction rate density:
\begin{equation}
  \begin{aligned}
    &\int_{\vt a}^{\vt b} \prod_{i \in \vt a} f_i^{(0)} \left[
      \abs{\pzcM_{\vt a}^{\vt b}}^2 \left( \prod_{i \in \vt a} \frac{n_i}{n_i^{(0)}} - \prod_{i \in \vt b} \frac{n_i}{n_i^{(0)}} \right)
      + \abs{\delta \pzcM_{\vt a}^{\vt b}}^2 \left( \prod_{i \in \vt b} \frac{n_i}{n_i^{(0)}} \right)
    \right] \\
    &= \left( \prod_{i \in \vt a} \frac{n_i}{n_i^{(0)}} - \prod_{i \in \vt b} \frac{n_i}{n_i^{(0)}} \right)
      \underbrace{\int_{\vt a}^{\vt b} \abs{\pzcM_{\vt a}^{\vt b}} \prod_{i \in \vt a} f_i^{(0)}}_{\gamma} \\
    &\quad + \left( \prod_{i \in \vt b} \frac{n_i}{n_i^{(0)}} \right) \underbrace{\int_{\vt a}^{\vt b} \abs{\delta \pzcM_{\vt a}^{\vt b}}^2 \prod_{i \in \vt a} f_i^{(0)}}_{\delta \gamma}
  \end{aligned}
\end{equation}

\subsection{\texorpdfstring{\(1 \leftrightarrow n\)}{1 to n} Interactions}

The reaction density is given by
\begin{equation}
  \left( \frac{n_a}{n_a^{(0)}} - \prod_{i \in \vt b} \frac{n_i}{n_i^{(0)}} \right) \int_{a}^{\vt b} \abs{\pzcM_a^{\vt b}}^2 f_a^{(0)}
\end{equation}

\begin{equation}
  \begin{aligned}
    \int_a^{bc} f_a^{(0)}
    &= \left[ \int \dd \Pi_a f_a^{(0)} \right] \left[ \int \dd \Pi_{\vt b} (2 \pi)^4 \delta^4(p_a - p_{\vt b}) \abs{\pzcM_a^{\vt b}} \right] \\
    &= \left[ g_a \frac{m_a K_1(m_a \beta)}{4 \pi^2 \beta} \right] \Big[ 2 m_a \Gamma(a \to \vt b) \Big]
  \end{aligned}
\end{equation}

\subsubsection{\texorpdfstring{\(1 \leftrightarrow 2\)}{1 to 2} Decay}

In the specific case of a \(1 \leftrightarrow 2\) interaction, the expression is
\begin{align}
  \left[ \frac{n_a}{n_a^{(0)}} - \frac{n_b n_c}{n_b^{(0)} n_c^{(0)}} \right] \int_a^{bc} \abs{\pzcM_a^{bc}} f_a^{(0)}.
\end{align}
For three particle interaction summed over all spins, all dependence on external
momenta vanish and thus the integral can be done exactly with the matrix element
outside of the integral
\begin{equation}
  \begin{aligned}
    \int_a^{bc} f_a^{(0)}
    &= \left[ \int \dd \Pi_a f_a^{(0)} \right] \left[ \int \dd \Pi_b \dd \Pi_c (2 \pi)^4 \delta^4(p_a - p_b - p_c) \right] \\
    &= \left[ g_a \frac{m_a K_1(m_a \beta)}{4 \pi^2 \beta} \right] \left[ g_b g_c \frac{\lambda^\frac{1}{2}(m_a^2, m_b^2, m_c^2)}{8 \pi m_a^2} \right] \\
    &= \frac{g_a g_b g_c}{32 \pi^3} \lambda^\frac{1}{2}(m_a^2, m_b^2, m_c^2) \frac{K_1(m_a \beta)}{m_a \beta}
  \end{aligned}
\end{equation}
Hence the interaction rate is
\begin{equation}
  \left[ \frac{n_a}{n_a^{(0)}} - \frac{n_b n_c}{n_b^{(0)} n_c^{(0)}} \right]
  \frac{g_a g_b g_c}{32 \pi^3} \lambda^\frac{1}{2}(m_a^2, m_b^2, m_c^2) \frac{K_1(m_a \beta)}{m_a \beta} \abs{\pzcM_a^{bc}}
\end{equation}

Although the above form is mathematically sound, it quickly becomes difficult to
compute numerically as \(\beta\) becomes larger since \(K_1(m_a \beta) \to 0\)
and \(n_a^{(0)} \sim K_2(m_a \beta) \to 0\) resulting in a \(0 / 0\) error.
Note that we never need to worry about the other equilibrium number densities
going to 0 as it is always the heaviest particle that is decaying.  Thus we
define:
\begin{equation}
  \begin{aligned}
    \tilde\gamma &\defeq \frac{\gamma}{n_a^{(0)}} \\
    &= \frac{g_b g_c \zeta(3)}{16 \pi^3} \lambda^{\frac{1}{2}}(m_a^2, m_b^2, m_c^2) \frac{K_1(m_a \beta)}{K_2(m_a \beta)} \frac{1}{(m_a \beta)^3} \abs{\pzcM_a^{bc}}^2
  \end{aligned}
\end{equation}
and the interaction rate therefore can be expressed as
\begin{equation}
  \left[ n_a - n_a^{(0)} \frac{n_b n_c}{n_b^{(0)} n_c^{(0)}} \right] \tilde \gamma
\end{equation}

The ratio of Bessel functions behaves as
\begin{equation}
  \frac{K_1(x)}{K_2(x)} = \begin{cases}
    \frac{x}{2} + \calO(x^3)                                              & x \ll 1 \\
    1 - \frac{3}{2x} + \frac{15}{8 x^2} + \calO\left(\frac{1}{x^3}\right) & x \gg 1
  \end{cases},
\end{equation}
and thus even as \(m_a \beta \to \infty\) the first particle will continue to
decay even if its equilibrium number density is 0 (as expected).

Once the normalization is taken into account, then \(\gamma \sim \beta\) as
\(\beta \gg m_a\) and the interaction rate keeps increasing so that the particle
will always decay eventually.

\subsubsection{\texorpdfstring{\(1 \leftrightarrow 3\)}{1 to 3} Decay}%
\label{sec:1-to-3_decay}

\todo{To calculate}

\subsection{\texorpdfstring{\(2 \leftrightarrow n\)}{1 to n} Scattering}%
\label{sec:2-to-n_scattering}

The reaction density is given by
\begin{equation}
  \left( \frac{n_a n_b}{n_a^{(0)} n_b^{(0)}} - \prod_{i \in \vt c} \frac{n_i}{n_i^{(0)}} \right) \int_{ab}^{\vt c} \abs{\pzcM_{ab}^{\vt c}}^2 f_a^{(0)} f_b^{(0)}.
\end{equation}

In order to make it easier to compute, we insert the identity from \cref{eq:phase_space_identity}
\begin{equation}
  \begin{aligned}
    \dd \Pi_a \dd \Pi_b &\equiv \dd \Phi_2(q_{ab}; p_a, p_b) \ddbar s \dd \Pi_{ab} \\
    &= \frac{g_a g_b}{8 \pi} \frac{\lambda^\frac{1}{2}(s, m_a^2, m_b^2)}{s} \ddbar s \dd\Pi_{ab}
  \end{aligned}
\end{equation}
Specifically, since we are assuming Maxwell--Boltzmann distributions, then
\(f_a^{(0)} f_b^{(0)} = \exp[-(E_a + E_b) \beta] = \exp[-E_{ab} \beta]\) and the
squared amplitude depends on \(s\) but not \(\vt p_{ab}\) so we can integrate
over the intermediate momentum we introduced:
\begin{equation}
  \begin{aligned}
    \int \dd \Pi_{ab} f_a^{(0)} f_b^{(0)}
    &= \int \ddbar^3{\vt p_{ab}} \frac{1}{2 E_{ab}} e^{- E_{ab} \beta} \\
    &= \int \frac{\sqrt{E_{ab}^2 - s}}{2} e^{- E_{ab} \beta} \frac{\dd E_{ab} \dd \Omega_{ab}}{(2 \pi)^3} \\
    &= \frac{1}{4 \pi^2} \frac{\sqrt{s}}{\beta} K_1(\sqrt{s} \beta)
  \end{aligned}
\end{equation}
Putting together the initial particle integration, we obtain:
\begin{equation}
  \begin{aligned}
    \int_{ab}^{\vt b} \abs{\pzcM_{ab}^{\vt c}} f_a^{(0)} f_b^{(0)}
    &= \frac{1}{4 \pi^2} \int \ddbar s \frac{\sqrt{s}}{\beta} K_1(\sqrt{s} \beta) \\
    &\qquad \times \frac{g_a g_b}{8 \pi} \frac{\lambda^\frac{1}{2}(s, m_a^2, m_b^2)}{s} \\
    &\qquad \times \int \dd \Pi_{\vt c} (2\pi)^4 \delta^4(q_{ab} - p_{\vt c}) \abs{\pzcM_{ab}^{\vt c}} \\
    &= \frac{g_a g_b}{64 \pi^4} \int \dd s \frac{K_1(\sqrt{s} \beta)}{\sqrt{s} \beta} \\
    &\qquad \times \lambda^\frac{1}{2}(s, m_a^2, m_b^2) \\
    &\qquad \times \int \dd \Pi_{\vt c} (2\pi)^4 \delta^4(q_{ab} - p_{\vt c}) \abs{\pzcM_{ab}^{\vt c}}
  \end{aligned}
\end{equation}

\subsubsection{\texorpdfstring{\(2 \leftrightarrow 2\)}{2 to 2} Scattering}%
\label{sec:2-to-2_scattering}

When both initial and final states have two particles, both initial and final
two-body phase spaces can be greatly simplified.  Specifically, the integration
over the final state particles is:
\begin{equation}
  \begin{aligned}
    \int \dd \Pi_c \dd \Pi_d (2 \pi)^4 \delta^4(q_{ab} - p_c - p_d) \abs{\pzcM_{ab}^{cd}}
    &= \int \dd \Phi_2(q_{ab}; p_c, p_d) \abs{\pzcM_{ab}^{cd}} \\
    &= \frac{g_c g_d}{32 \pi^2} \frac{\lambda^\frac{1}{2}(s, m_c^2, m_d^2)}{s} \int \dd \Omega_{cd} \abs{\pzcM_{ab}^{cd}}
  \end{aligned}
\end{equation}
and the integration over \(\phi_{cd}\) can be done trivially and the
integration over \(\cos \theta_{cd}\) can be done by relating it to the
Mandelstam \(t\) variable (\cref{eq:t_angular})
\begin{equation}
  \begin{aligned}
    &= \frac{g_c g_d}{32 \pi^2} \frac{\lambda^\frac{1}{2}(s, m_c^2, m_d^2)}{s} \int 2 \pi \frac{\dd \cos\theta}{\dd t} \dd t \abs{\pzcM_{ab}^{cd}} \\
    &= \frac{g_c g_d}{8 \pi} \frac{1}{\lambda^\frac{1}{2}(s, m_a^2, m_b^2)} \int \dd t \abs{\pzcM_{ab}^{cd}}
  \end{aligned}
\end{equation}

Thus the whole interaction rate is
\begin{equation}
  \begin{aligned}
    \int_{ab}^{cd} \abs{\pzcM_{ab}^{cd}} f_a^{(0)} f_b^{(0)}
    &= \frac{g_a g_b g_c g_d}{512 \pi^5} \int \frac{K_1(\sqrt s \beta)}{\sqrt s \beta} \abs{\pzcM_{ab}^{cd}} \dd t \dd s
  \end{aligned}
\end{equation}

As with the decay case, it can be numerically more stable to compute the
interaction rate \(\gamma\) divided by the (normalized) number density; however,
unlike the decay case there are multiple cases to consider depending on the
number of heavy particles with near-zero equilibrium number density:
\begin{subequations}
  \begin{align}
    \frac{K_1(\sqrt s \beta)}{\sqrt s \beta} \frac{1}{n_a^{(0)}}
    &= \left( \sqrt{\frac{8}{\pi}} \zeta(3) \right) \frac{\sqrt{\pi / 2}}{g_a} \frac{e^{\beta (m_a - \sqrt{s})}}{(\sqrt{m_a} \sqrt[4]{s} \beta)^3} \\
    \frac{K_1(\sqrt s \beta)}{\sqrt s \beta} \frac{1}{n_a^{(0)} n_b^{(0)}}
    &= \left( \sqrt{\frac{8}{\pi}} \zeta(3) \right)^2 \frac{\sqrt{\pi / 2}}{g_a g_b} \frac{e^{\beta (m_a + m_b - \sqrt{s})}}{\left(\sqrt{m_a m_b} \sqrt[4]{s} \beta^\frac{3}{2}\right)^3} \\
    \frac{K_1(\sqrt s \beta)}{\sqrt s \beta} \frac{1}{n_a^{(0)} n_b^{(0)} n_c^{(0)}}
    &= \left( \sqrt{\frac{8}{\pi}} \zeta(3) \right)^3 \frac{\sqrt{\pi / 2}}{g_a g_b g_c} \frac{e^{\beta (m_a + m_b + m_c - \sqrt{s})}}{\left(\sqrt{m_a m_b m_c} \sqrt[4]{s} \beta^2\right)^3} \\
    \frac{K_1(\sqrt s \beta)}{\sqrt s \beta} \frac{1}{n_a^{(0)} n_b^{(0)} n_c^{(0)} n_d^{(0)}}
    &= \left( \sqrt{\frac{8}{\pi}} \zeta(3) \right)^4 \frac{\sqrt{\pi / 2}}{g_a g_b g_c g_d} \frac{e^{\beta (m_a + m_b + m_c + m_d - \sqrt{s})}}{\left(\sqrt{m_a m_b m_c m_d} \sqrt[4]{s} \beta^\frac{5}{2} \right)^3}
  \end{align}
\end{subequations}
Important to note is that as \(s \to \infty\), all of the above are
exponentially suppressed, but in the case of 3 and 4 heavy particles there is a
partial exponential enhancement when \(\max\{m_a + m_b, m_c +
m_d\} < \sqrt{s} < m_a + m_b + m_c (+ m_d)\).

In the special case where \(\abs{\pzcM_{ab}^{cd}}\) is constant and only one of
the four particles is massive, the interaction rate can be evaluated exactly to
be
\begin{equation}
  \int_{ab}^{cd} \abs{\pzcM_{ab}^{cd}} f_a^{(0)} f_b^{(0)}
  = \frac{g_a g_b g_c g_d}{128 \pi^5} \frac{m K_1(m \beta)}{\beta^3} \abs{\pzcM_{ab}^{cd}}
\end{equation}
After normalization by \(1 / H \beta N_1\)

\subsubsection{Mandelstam Variables}%
\label{sec:mandelstam_variables}

Note that in the centre-of-mass frame of reference, we have:
\begin{align}
  p_a & = (E_a, \abs{\vt p}, 0, 0)                                       & p_b & = (E_b, -\abs{\vt p}, 0, 0) \\
  p_c & = (E_c, \abs{\vt p} \cos \theta, 0, \abs{\vt p} \sin \theta)     &
  p_d & = (E_c, - \abs{\vt p} \cos \theta, 0, - \abs{\vt p} \sin \theta)
\end{align}
and the Mandelstam variables
\begin{align}
  s & = (p_a + p_b)^2 = (p_c + p_d)^2 \\
  t & = (p_a - p_c)^2 = (p_b - p_d)^2 \\
  u & = (p_a - p_d)^2 = (p_b - p_c)^2
\end{align}
By requiring everything to be on-shell, we can solve all the energies and
\(\abs{\vt p}\) in terms of only masses and \(s\):
\begin{align}
  E_a & = \frac{s + m_a^2 - m_b^2}{2 \sqrt{s}} &
  E_b & = \frac{s + m_b^2 - m_a^2}{2 \sqrt{s}}   \\
  E_c & = \frac{s + m_c^2 - m_d^2}{2 \sqrt{s}} &
  E_d & = \frac{s + m_d^2 - m_c^2}{2 \sqrt{s}}   \\
\end{align}
\begin{equation}
  \begin{aligned}
    \abs{\vt p}   & = \frac{\lambda^{\frac{1}{2}}(s, m_a^2, m_b^2)}{2 \sqrt{s}} = \frac{\lambda^{\frac{1}{2}}(s, m_c^2, m_d^2)}{2 \sqrt{s}} \\
    \abs{\vt p}^2 & = \frac{\lambda^{\frac{1}{2}}(s, m_a^2, m_b^2) \lambda^{\frac{1}{2}}(s, m_c^2, m_d^2)}{4 s}
  \end{aligned}
\end{equation}

Now the \(t\) Mandelstam variable is:
\begin{align}
  t & = (E_a - E_c)^2 - 2 \abs{\vt p}^2 + 2 \abs{\vt p}^2 \cos \theta                                                \\
    & = \frac{\left( m_a^2 - m_b^2 - m_c^2 + m_d^2 \right)^2}{4 s}
  - \frac{\lambda^{\frac{1}{2}}(s, m_a^2, m_b^2) \lambda^{\frac{1}{2}}(s, m_c^2, m_d^2)}{2 s}                        \\
    & \qquad + \frac{\lambda^{\frac{1}{2}}(s, m_a^2, m_b^2) \lambda^{\frac{1}{2}}(s, m_c^2, m_d^2)}{2 s} \cos \theta
\end{align}
And in particular, one can relate a differential in \(\cos \theta\) to a differential in \(t\):
\begin{equation}
  \label{eq:t_angular}
  \ddfrac{t}{\cos \theta} = 2 \abs{\vt p}^2 = \frac{\lambda^{\frac{1}{2}}(s, m_a^2, m_b^2) \lambda^{\frac{1}{2}}(s, m_c^2, m_d^2)}{2 s}
\end{equation}

As for the domain of integration of \(t\), it is not difficult to see that
\(\theta \in [0, \pi]\) corresponds to
\begin{equation}
  t \in \left[
    \frac{\left( m_a^2 - m_b^2 - m_c^2 + m_d^2 \right)^2}{4 s},
    \frac{\left( m_a^2 - m_b^2 - m_c^2 + m_d^2 \right)^2}{4 s}
    - \frac{\lambda^{\frac{1}{2}}(s, m_a^2, m_b^2) \lambda^{\frac{1}{2}}(s, m_c^2, m_d^2)}{s}
    \right]
\end{equation}

\subsection{Chemical Potentials}
\label{sec:chemical_potentials}

Provided that chemical potentials are small, 

\cleardoublepage
\section{Boltzmann Equations}%
\label{sec:boltzmann_equation}

We consider an interaction \(ab \leftrightarrow cd\), by \gls{CPT}
invariance we have that:
\begin{equation}
  \gamma(ab \to cd) = \gamma(\overline{c} \overline{d} \to \overline{a} \overline{b})
\end{equation}
and the \gls{CP} asymmetry is taken into account with \(\epsilon\) such
that
\begin{equation}
  \gamma(ab \to cd) - \gamma(\overline{a} \overline{b} \to \overline{c} \overline{d})
  = \epsilon
  = \gamma(\overline{c} \overline{d} \to \overline{a} \overline{b}) - \gamma(cd \to ab).
\end{equation}
and using \gls{CPT} invariance as well, this can be written as
\begin{equation}
  \delta\gamma = \gamma(ab \to cd) - \gamma(cd \to ab).
\end{equation}
As a result, any forward or backward process and their \gls{CP}
conjugate can expressed in terms of the original forward rate \(\gamma\), and
the asymmetry \(\epsilon\).

We will also be defined the asymmetry between the particle and anti-particle
number densities
\begin{equation}
  \Delta_a \defeq n_a - n_{\overline a}.
\end{equation}
and it is worth noting that generally
\begin{equation}
  \abs{\delta\gamma} \ll \gamma \qquad\qquad \abs{\Delta_a} \ll n_a,
\end{equation}
and in the \gls{CP}-conserving limit, both \(\epsilon\) and \(\Delta_a\)
go to zero.

The change in number density for \(a\) and \(\overline{a}\) are:
\begin{align}
  \ddfrac{n_a}{t}
   & = \frac{n_c n_d}{n_c^{(0)} n_d^{(0)}} \gamma(cd \to ab)
  - \frac{n_a n_b}{n_a^{(0)} n_b^{(0)}} \gamma(ab \to cd)                                                                      \\
   & = \left[
  \frac{n_c n_d}{n_c^{(0)} n_d^{(0)}}
  - \frac{n_a n_b}{n_a^{(0)} n_b^{(0)}}
  \right] \gamma
  + \frac{n_c n_d}{n_c^{(0)} n_d^{(0)}} \epsilon,                                                                              \\
  \ddfrac{n_{\overline a}}{t}
   & = \frac{n_{\overline c} n_{\overline d}}{n_c^{(0)} n_d^{(0)}} \gamma(\overline c \overline d \to \overline a \overline b)
  - \frac{n_{\overline a} n_{\overline b}}{n_a^{(0)} n_b^{(0)}} \gamma(\overline a \overline b \to \overline d \overline c)    \\
   & = \left[
  \frac{n_{\overline c} n_{\overline d}}{n_c^{(0)} n_d^{(0)}}
  - \frac{n_{\overline a} n_{\overline b}}{n_a^{(0)} n_b^{(0)}}
  \right] \gamma
  + \frac{n_{\overline a} n_{\overline b}}{n_a^{(0)} n_b^{(0)}} \delta\gamma.                                                      \\
\end{align}

In the \gls{CP} conserving limit, both of these simplify to
\begin{equation}
  \ddfrac{n_a}{t} = \ddfrac{n_{\overline a}}{t}
  = \left[ \frac{n_c n_d}{n_c^{(0)} n_d^{(0)}}  - \frac{n_a n_b}{n_a^{(0)} n_b^{(0)}} \right] \gamma(ab \to cd)
\end{equation}

Computing the rate of change of the \gls{CP} asymmetry (obviously before
taking any limits), we obtain:
\begin{equation}
  \begin{aligned}
    \ddfrac{\Delta_a}{t}
     & = \ddfrac{n_a}{t} - \ddfrac{n_{\overline a}}{t} \\
     & = \left[
    \frac{\Delta_c n_d + \Delta_d n_c}{n_c^{(0)} n_d^{(0)}}
    - \frac{\Delta_a n_b + \Delta_b n_a}{n_a^{(0)} n_b^{(0)}} \right] \gamma
    - \left[
      \frac{n_c n_d}{n_c^{(0)} n_d^{(0)}} - \frac{n_a n_b}{n_a^{(0)} n_b^{(0)}}
    \right] \delta\gamma
  \end{aligned}
\end{equation}
where we have omitted terms proportional to \(\Delta^2\) and \(\delta\gamma \Delta\)
(for any combination of \(\Delta_i\)).

\subsection{Three-Body Interactions}%
\label{sec:three-body_interactions}

Given a three-body interaction with (tree-level) squared amplitude
\(\abs{\pzcM}^2\), then there are six possible interactions (ignoring their
\gls{CPT} counterparts):
\begin{equation}
  \begin{aligned}
    a           & \to bc                         & \qquad
    \overline b & \to \overline ac               & \qquad
    \overline c \to \overline ab                          \\
    a           & \gets bc                       &
    \overline b & \gets \overline ac
                & \overline c \gets \overline ab
  \end{aligned}
\end{equation}
The number density for \(a\) therefore is:
\begin{equation}
  \begin{aligned}
    \ddfrac{n_a}{t} =
     & - \left( \frac{n_a}{n_a^{(0)}} - \frac{n_b}{n_b^{(0)}} \frac{n_c}{n_c^{(0)}} \right) \gamma(a \leftrightarrow bc)                       \\
     & + \left( \frac{n_b}{n_b^{(0)}} - \frac{n_a}{n_a^{(0)}} \frac{n_c}{n_c^{(0)}} \right) \gamma(\overline b \leftrightarrow \overline a c)  \\
     & + \left( \frac{n_c}{n_c^{(0)}} - \frac{n_a}{n_a^{(0)}} \frac{n_b}{n_b^{(0)}} \right) \gamma(\overline c \leftrightarrow \overline a b).
  \end{aligned}
\end{equation}

\subsection{Asymmetry Evolution}%
\label{sec:asymmetry_evolution}

We define the asymmetry in the phase space and number densities as
\begin{align}
  \Delta f_a & \defeq f_a - f_{\overline a}, &
  \Delta n_a & \defeq n_a - n_{\overline a},
\end{align}
and taking \(\mu_a = - \mu_{\overline a}\) (which is valid provided
interactions \(a \overline a \leftrightarrow (\dots)\) are fast), then we have
that
\begin{align}
  \Delta f_a & \defeq f_a^{(0)} \left( e^{-\mu \beta} - e^{\mu \beta} \right), &
  \Delta n_a & \defeq n_a^{(0)} \left( e^{-\mu \beta} - e^{\mu \beta} \right).
\end{align}
As a result, the asymmetry in the phase space can the related to the number
density asymmetry through
\begin{equation}
  \Delta f_a = f_a^{(0)} \left( e^{-\mu \beta} - e^{\mu \beta} \right)
  = f_a^{(0)} \frac{\Delta n_a}{n_a^{(0)}}
\end{equation}

The rate of change in the number density asymmetry is
\begin{equation}
  \begin{aligned}
    \ddfrac{\Delta n_a}{t}
     & = - \int_{\vt a}^{\vt b} \abs{\pzcM}^2 \left[ \prod_{i \in \vt a} \Delta f_i - \prod_{i \in \vt b} \Delta f_i \right]                                                         \\
     & = - \int_{\vt a}^{\vt b} \abs{\pzcM}^2 \left[ \prod_{i \in \vt a} \frac{\Delta n_i}{n_i^{(0)}} f_i^{(0)} - \prod_{i \in \vt b} \frac{\Delta n_i}{n_i^{(0)}} f_i^{(0)} \right] \\
     & = - \left[ \gamma(\vt a \to \vt b) \prod_{i \in \vt a} \frac{\Delta n_i}{n_i^{(0)}} - \gamma(\vt b \to \vt a) \prod_{i \in \vt b} \frac{\Delta n_i}{n_i^{(0)}} \right]
  \end{aligned}
\end{equation}
On the other hand, one can also look at the change in the number density
asymmetries from looking at the change in the number densities of the particle
and antiparticle respectively:
\begin{equation}
  \begin{aligned}
    \ddfrac{\Delta n_a}{t}
     & = \ddfrac{n_a}{t} - \ddfrac{n_{\overline a}}{t}                                                                                                                                                                                  \\
     & = - \left[ \gamma(\vt a \to \vt b) \prod_{i \in \vt a} \frac{n_i}{n_i^{(0)}} - \gamma(\vt b \to \vt a) \prod_{i \in \vt b} \frac{n_i}{n_i^{(0)}} \right]                                                                         \\
     & \quad + \left[ \gamma(\overline{\vt a} \to \overline{\vt b}) \prod_{i \in \vt a} \frac{n_{\overline i}}{n_i^{(0)}} - \gamma(\overline{\vt b} \to \overline{\vt a}) \prod_{i \in \vt b} \frac{n_{\overline i}}{n_i^{(0)}} \right] \\
     & = - \gamma(\vt a \to \vt b) \left[ \prod_{i \in \vt a} \frac{n_i}{n_i^{(0)}} + \prod_{i \in \vt b} \frac{n_{\overline i}}{n_i^{(0)}} \right]
    - \gamma(\vt b \to \vt a) \left[ \prod_{i \in \vt a} \frac{n_{\overline i}}{n_i^{(0)}} + \prod_{i \in \vt b} \frac{n_i}{n_i^{(0)}} \right]
  \end{aligned}
\end{equation}
where we have used \gls{CPT} invariance to equal the rates \(\vt a \to
\vt b\) and \(\overline{\vt b} \to \overline{\vt a}\).  Further assuming
\gls{CP} symmetry, this can be simplified further to
\begin{equation}
  \ddfrac{\Delta n_a}{t} = - \gamma(\vt a \leftrightarrow \vt b) \left[ \prod_{i \in \vt a} \frac{n_i}{n_i^{(0)}} + \prod_{i \in \vt a} \frac{n_{\overline i}}{n_i^{(0)}} + \prod_{i \in \vt b} \frac{n_{\overline i}}{n_i^{(0)}} + \prod_{i \in \vt b} \frac{n_i}{n_i^{(0)}} \right]
\end{equation}

\subsection{Number Density Prefactors}%
\label{sec:number_density_prefactors}

For a three-particle integration, the prefactor to account for the number
densities in the forward and backward rates are:
\begin{equation}
  \begin{aligned}
    p(1 \to 2)
    &= \int_{\vt a}^{\vt b} f_a = \int \dd \Pi_a f_a \\
    &= e^{\mu \beta} \frac{m K_1(m \beta)}{4 \pi^2 \beta}
  \end{aligned}
\end{equation}

\printglossaries%


\end{document}
